\section{Functors} % (fold)
\label{sec:functors}
A category is itself a type of mathematical structure and so, one can generalise the notion of a morphism thus preserve this structure by the notion of a
functor.
A functor associates to every object of one category an object of another category, and to every morphism in the first category a morphism in the second.
Hence, functors are structure-preserving maps between categories and can be thought of as morphisms in the category of all (small) categories.

In particular, what we have done is define a category of categories and functors - the objects are categories, and the morphisms (between categories) 
are functors.
By studying categories and functors, we are not merely studying a class of mathematical structures and the morphisms between them,
we are studying the relationships between various classes of mathematical structures.

\begin{defn}[Functor]
	Let $\mathcal{C}$ and $\mathcal{K}$ be categories. A $\emph{functor}$ $F$ from $\mathcal{C}$ to $\mathcal{K}$ is a mapping that:
	\begin{enumerate}
		\item associates to each object $X \in \mathcal{C}$ an object $F(X) \in  \mathcal{K}$
		
		\item associates to each morphism $\varphi: X \to Y \in \mathcal{C}$ a morphism $F(\varphi) : F(X) \to F(Y) \in \mathcal{K}$
		satisfying:
		\begin{enumerate}
			\item $F(id_{X}) = id_{F(X)}$ for every object $X \in \mathcal{C}$
			
			\item $F(\psi \circ \varphi) = F(\psi) \circ F(\varphi)$ for all morphisms $\varphi: X \to Y$ and $\psi: Y \to Z$
			\begin{rem}
				That is, functors must preserve identity morphisms and composition of morphisms.
			\end{rem}
		\end{enumerate}
	\end{enumerate}
\end{defn}

\subsection{Types of Functors} % (fold)
\label{subsec:functorstypes}
Like many things in category theory, functors come in a kind of ``dual'' type in concepts;
that of the $\emph{contravariant}$ and $\emph{covariant}$ functors, defined as follows:

\begin{defn}[Covariant Functor]
 A $\emph{covariant}$ functor $F$ from a category $\mathcal{C}$ to a category $\mathcal{K}$, $F: \mathcal{C} \to \mathcal{K}$, consists of:
 \begin{itemize}
  \item for each object $X \in \mathcal{C}$, an object $F(X) \in \mathcal{K}$
  \item for each morphism $\varphi: X \to Y \in \mathcal{C}$, a morphism $F(\varphi): F(X) \to F(Y)$
 \end{itemize}
 provided the following two properties hold:
 \begin{itemize}
  \item For every object $X \in \mathcal{C}$, $F(\Id{X}) = \Id{F(X)}$
  \item For all morphisms $\varphi: X \to Y$ and $\psi: Y \to Z$, $F(\psi \circ \varphi) = F(\psi) \circ F(\varphi)$
 \end{itemize}
\end{defn}

\begin{defn}[Contravariant Functor]
 A $\emph{contravariant}$ functor $F: \mathcal{C} \to \mathcal{K}$, is a `reversed' covariant functor.
 In particular, for every morphism $\varphi: X \to Y \in \mathcal{C}$ must be assigned to a morphism $F(\varphi): F(Y) \to F(X) \in \mathcal{K}$.
 Alternatively, contravariant functors act as covariant functors from the opposite category $\mathcal{C}^{op} \to \mathcal{K}$.
\end{defn}

Functors have particular properties that essentially are the same as morphisms.
This may seem obvious given that functors are a generalisation of the notion of a morphism.
So, the following definitions should mostly be obvious if you reflect back to the corresponding morphism
definitions above or simply break the word down grammatically.

\begin{defn}[Endofunctor]
 A $\emph{endofunctor}$ is a functor that maps a category to itself, sometimes called a $\emph{identity functor}$.
\end{defn}

\begin{defn}[Bifunctor]
 A $\emph{bifunctor}$ is a functor in $\emph{two}$ arguments. More formally, a bifunctor is a functor whose
 domain is a $\emph{product category}$.
 The $Hom$ functor is a natural example; it is contravariant in one argument while covariant in the other.
 Hence, the $Hom$ functor is of the type $\mathcal{C}^{op} \times \mathcal{C} \to \textbf{Set}$.
\end{defn}

\begin{defn}[Multifunctor]
 A $\emph{multifunctor}$ is a generalisation of the functor concept to n variables, (e.g., A $\emph{bifunctor}$ is when $n=2$)
\end{defn}

\subsection{Properties of Functors} % (fold)
\label{subsec:functorsproperties}
However, some properties are more particular to functors and we review them here to families ourselves.
A functor $F : \mathcal{A} \to \mathcal{B}$ is
\begin{enumerate}
 \item $\emph{faithful}$ if for every parallel pair of morphisms $f,g : A \rightrightarrows A' \in \mathcal{A}$, one has $f = g$ whenever $F(f) = F(g)$.
 (Recall the definition of a monomorphism).
 \item $\emph{full}$ if for every morphism $b: F(A) \to F(A') \in \mathcal{B}$, there exists a morphism $a: A \to A' \in \mathcal{A}$ such that $F(a) = b$
 (Recall the definition of a epimorphism).
 \item essentially $\emph{surjective}$ if for every object $B \in \mathcal{B}$, there exists an object $A \in \mathcal{A}$ with $B$ isomorphic to $F(A)$
 \item an $\emph{equivalence}$ if there exists a functor $F': \mathcal{B} \to \mathcal{A}$ such that both $F \circ F'$ and $F' \circ F$ are naturally
 isomorphic to the identity functors; such a functor $F'$ is called a $\emph{quasi-inverse}$ of $F$
 \item an $\emph{isomorhpism}$ if there exists a functor $F': \mathcal{B} \to \mathcal{A}$ such that both $F \circ F'$ and $F' \circ F$
 are equal to the identity functors
 \item $\emph{conserative}$ if it reflects isomorphisms; that is, $a: A \to A'$ is an isomorphism whenever $F(a): F(A) \to F(A')$ is
\end{enumerate}

\begin{defn}[Functor category and the Yoneda embedding]
The Yoneda lemma suggests that instead of studying the $\emph{locally small}$ category $\mathcal{K}$,
one should study the category of all functors of $\mathcal{K}$ into $\textbf{Set}$.
Since the category $\textbf{Set}$ is well understood, a functor of $\mathcal{K}$ into $\textbf{Set}$
maybe seen as a $\emph{representation}$ of $\mathcal{K}$ in terms of known structures.
See the $\emph{category of diagrams}$ later.
 \begin{enumerate}
  \item Given a category $\mathcal{K}$ and a small category $\mathcal{J}$, we denote by
  $\mathcal{K}^\mathcal{J}$ the category of functors from $\mathcal{J} \to \mathcal{K}$ and natural transformations
  of each functor from $\mathcal{J} \to \mathcal{K}$.
  \item In case $\mathcal{K} = \textbf{Set}$, we have the $\emph{Yoneda embedding}$:\\
  $Y_{\mathcal{J}}: \mathcal{J}^{op} \to \textbf{Set}^{\mathcal{J}}
  \, \,
  Y_{\mathcal{J}}(X) = \mathcal{J}(X, -)$, which is full and faithful (Recall the definition of a bimorphism).
 \end{enumerate}
\end{defn}


\subsection{F-algebra} % (fold)
\label{subsec:f-algebra}
$\emph{F-algebras}$ have various applications, such as in computer science,
creating concrete definitions of various algebraic data types (e.g., lists, trees, ...).
A $F$-algebra is a structure defined according to some endofunctor $F: \mathcal{K} \to \mathcal{K}$
for some category of functors $\mathcal{K}$. That is,

\begin{defn}[F-algebra]
 A $F$-algebra for a endofunctor $F$ is some object $A$ in the category $\mathcal{K}$, coupled with
 a $\mathcal{K}$-morphism $\alpha: F(A) \to A$. We denote this $F$-algebra by the pair $(A, \alpha)$.
\end{defn}

$F$-algebras also form a category as expected.
Since, The homomorphism from a $F$-algebra $(A, \alpha)$ to $F$-algebra $(B, \beta)$
is given by $\varphi: A \to B \in \mathcal{K}$ such that $\varphi \circ \alpha = \beta \circ F(\varphi)$.
Hence, we have the following commutative diagram:
\begin{tikzcd}
F(A) \rar[][color=red]{\alpha} \dar[]{F(\varphi)}
 & A \dar[][color=red]{\varphi} \\
F(B) \rar{\beta}
 & B
\end{tikzcd}

\begin{exmp}[Initial algebra]
 Consider a functor $F: \textbf{Set} \to \textbf{Set}$, in particular, $F: X \to 1 \bigoplus X$.
 Then the set $\N$ endowered with the function $[zero,succ]: 1 \bigoplus \N \to N$, is a $F$-algebra.
 This initial algebra is denoted by the pair $(\N, [zero,succ])$.
\end{exmp}

Here $1$ denotes the $\emph{terminal object}$ taken to be the singleton set $\{ \; \}$
and $\bigoplus$ as the $\emph{coproduct}$.

\begin{rem}
 The categorial dual are called $F$-$\emph{co}$algebras.
\end{rem}