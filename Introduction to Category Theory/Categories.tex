% Copyright © 2012 Edward O'Callaghan. All Rights Reserved.

\RequirePackage[l2tabu, orthodox]{nag}

\documentclass[10pt, oneside, reqno]{amsart}
\usepackage{geometry, setspace, graphicx, enumerate, amssymb}
%\usepackage{dcpic,pictex}
\usepackage{pgf}
\usepackage{tikz}
\usepackage{tikz-cd}
%\usepackage{bbm} % /mathbbm{1}
\usepackage{microtype}
\onehalfspacing                 

%\usepackage[ruled,section]{algorithm}
%\usepackage{algpseudocode}


% AMS Theorems
\theoremstyle{plain}% default 
\newtheorem{thm}{Theorem}[section] 
\newtheorem{prob}[thm]{Problem}
\newtheorem{question}[thm]{Question}

\newtheorem{lem}[thm]{Lemma}
\newtheorem{prop}[thm]{Proposition}
\newtheorem*{cor}{Corollary}

\theoremstyle{definition}
\newtheorem{defn}[thm]{Definition}
\newtheorem{conj}[thm]{Conjecture}
\newtheorem{exmp}[thm]{Example}

\theoremstyle{remark} 
\newtheorem*{rem}{Remark} 
\newtheorem*{note}{Note} 
\newtheorem{case}{Case} 

\newcommand{\expc}[1]{\mathbb{E}\left[#1\right]}

\newcommand{\Q}{\mathbb{Q}}
\newcommand{\R}{\mathbb{R}}
\newcommand{\C}{\mathbb{C}}
\newcommand{\Z}{\mathbb{Z}}
\newcommand{\F}{\mathbb{F}}
\newcommand{\Ga}{\mathbb{G}}

\newcommand{\Cat}{\mathcal{K}}

\newcommand{\nth}{n\textsuperscript{th}}
\newcommand{\bigo}[1]{\mathcal{O}(#1)}

% \setlength{\topmargin}{0.2cm}
% \setlength{\footskip}{0.2cm}
% \setlength{\hoffset}{-1cm}
% \setlength{\voffset}{-2cm}

        
\usepackage{hyperref}
        
\title{Category Theory}                               % Document Title
\author{Edward O'Callaghan}
%\date{}                                           % Activate to display a given date or no date



\begin{document}
\maketitle \tableofcontents \clearpage


\section{Introduction} % (fold)
\label{sec:introduction}
We begin by defining what we mean by a `Category'.
\begin{defn}[Category]
	A $\emph{category}$ $\Cat$ consists of the following three mathematical entities:
	\begin{enumerate}
		\item A $\emph{class}$ $\text{Ob}(\Cat)$ of objects
		
		\item A class $\text{Hom}(A,B)$ of $\emph{morphisms}$, from $A \longrightarrow B$ such that $A, B \in \text{Ob}(\Cat)$.
		\\
		e.g. $f : A \to B$ to mean $f \in \text{Hom}(A,B)$.
		\begin{rem}
		 The class of \emph{all} morphisms of $\Cat$ is denoted $\text{Hom}(\Cat)$.
		\end{rem}

		\item Given $A, B, C \in \text{Ob}(\Cat)$, a binary operation $\circ : Hom(B,C) \times Hom(A,B) \to Hom(A,C)$ called \emph{composition},
		satisfying:
		\begin{enumerate}
			\item \emph{(associativity)} Given $f : A \to B$, $g : B \to C$ and $h : C \to D$
			\\
			we have $h \circ (g \circ f) = (h \circ g) \circ f$.
			\\
			\begin{tikzcd}[row sep=tiny]
			& B \arrow{dd}{g} \\
			A \arrow{ur}{f} \arrow{dr}{h} & \\
			& C
			\end{tikzcd}
			
			\item \emph{(identity)} For any object X there is an identity morphism $1_{X} : X \to X$ such that for any $f : A \to B$ we have $1_{B} \circ f = f = f \circ 1_{A}$.
			\\
			\begin{tikzcd}
			X \arrow[in=30, out=60, loop]{}[name=idX]{1_{X}}
			\end{tikzcd}
		\end{enumerate}
	\end{enumerate}
\end{defn}

It is also worth noting about what we mean by `small' and `large' categories.

\begin{defn}[Small Category]
	A category $\Cat$ is called small if both $\text{Ob}(\Cat)$ and $\text{Hom}(\Cat)$ are sets.
	If $\Cat$ is not small, then it is called large.
	$\Cat$ is called locally small if $\text{Hom}(A,B)$ is a set for all $A, B \in \text{Ob}(\Cat)$.
\end{defn}

\begin{rem}
	Most important categories in mathematics are not small however, are locally small.
\end{rem}


\end{document}% End of document.
