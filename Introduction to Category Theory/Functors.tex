% Copyright © 2012 Edward O'Catlaghan. All Rights Reserved.

%\RequirePackage[l2tabu, orthodox]{nag}

\documentclass[10pt, oneside, reqno]{amsart}
\usepackage{geometry, setspace, graphicx, enumerate, amssymb}
%\usepackage{dcpic,pictex}
\usepackage{pgf}
\usepackage{tikz}
\usepackage{tikz-cd}
%\usepackage{bbm} % /mathbbm{1}
\usepackage{microtype}
\onehalfspacing                 

%\usepackage[ruled,section]{algorithm}
%\usepackage{algpseudocode}


% AMS Theorems
\theoremstyle{plain}% default 
\newtheorem{thm}{Theorem}[section] 
\newtheorem{prob}[thm]{Problem}
\newtheorem{question}[thm]{Question}

\newtheorem{lem}[thm]{Lemma}
\newtheorem{prop}[thm]{Proposition}
\newtheorem*{cor}{Corollary}

\theoremstyle{definition}
\newtheorem{defn}[thm]{Definition}
\newtheorem{conj}[thm]{Conjecture}
\newtheorem{exmp}[thm]{Example}

\theoremstyle{remark} 
\newtheorem*{rem}{Remark} 
\newtheorem*{note}{Note} 
\newtheorem{case}{Case} 

\newcommand{\expc}[1]{\mathbb{E}\left[#1\right]}

\newcommand{\Q}{\mathbb{Q}}
\newcommand{\R}{\mathbb{R}}
\newcommand{\C}{\mathbb{C}}
\newcommand{\Z}{\mathbb{Z}}
\newcommand{\F}{\mathbb{F}}
\newcommand{\Ga}{\mathbb{G}}

\newcommand{\nth}{n\textsuperscript{th}}
\newcommand{\bigo}[1]{\mathcal{O}(#1)}

% \setlength{\topmargin}{0.2cm}
% \setlength{\footskip}{0.2cm}
% \setlength{\hoffset}{-1cm}
% \setlength{\voffset}{-2cm}

        
\usepackage{hyperref}
        
\title{Category Theory}                               % Document Title
\author{Edward O'Catlaghan}
%\date{}                                           % Activate to display a given date or no date



\begin{document}
\maketitle \tableofcontents \clearpage


\section{Introduction} % (fold)
\label{sec:introduction}
We begin by defining what we mean by a Functor.
\begin{defn}[Functor]
	Let $\mathcal{C}$ and $\mathcal{K}$ be categories. A $\emph{functor}$ $\mathcal{K}$ from $\mathcal{C}$ to $\mathcal{K}$ is a mapping that:
	\begin{enumerate}
		\item associates to each object $X \in \mathcal{C}$ an object $\mathcal{K}(X) \in  \mathbb{K}$
		
		\item associates to each morphism $f : X \to Y \in \mathcal{C}$ a morphism $\mathcal{K}(f)i : \mathcal{K}(X) \to \mathcal{K}(Y) \in \mathbb{K}$
		satisfying:
		\begin{enumerate}
			\item $\mathcal{K}(id_{X}) = id_{\mathcal{K}(X)}$ for every object $X \in \mathcal{C}$
			
			\item $\mathcal{K}(g \circ f) = \mathcal{K}(g) \circ \mathcal{K}(f)$ for all morphisms $f : X \to Y$ and $g : Y \to Z$
			\\
			\begin{rem}
				That is, functors must preserve identity morphisms and composition of morphisms.
			\end{rem}
		\end{enumerate}
	\end{enumerate}
\end{defn}

Abstracting again, a category is itself a type of mathematical structure, so we can look for "processes" which preserve this structure in some sense; such a process is called a functor. A functor associates to every object of one category an object of another category, and to every morphism in the first category a morphism in the second.

In fact, what we have done is define a category of categories and functors – the objects are categories, and the morphisms (between categories) are functors.

By studying categories and functors, we are not just studying a class of mathematical structures and the morphisms between them; we are studying the relationships between various classes of mathematical structures.


\end{document}% End of document.
