% Copyright © 2012 Edward O'Catlaghan. All Rights Reserved.

%\RequirePackage[l2tabu, orthodox]{nag}

\documentclass[10pt, oneside, reqno]{amsart}
\usepackage{geometry, setspace, graphicx, enumerate, amssymb}
%\usepackage{dcpic,pictex}
\usepackage{pgf}
\usepackage{tikz}
\usepackage{tikz-cd}
%\usepackage{bbm} % /mathbbm{1}
\usepackage{microtype}
\onehalfspacing                 

%\usepackage[ruled,section]{algorithm}
%\usepackage{algpseudocode}


% AMS Theorems
\theoremstyle{plain}% default 
\newtheorem{thm}{Theorem}[section] 
\newtheorem{prob}[thm]{Problem}
\newtheorem{question}[thm]{Question}

\newtheorem{lem}[thm]{Lemma}
\newtheorem{prop}[thm]{Proposition}
\newtheorem*{cor}{Corollary}

\theoremstyle{definition}
\newtheorem{defn}[thm]{Definition}
\newtheorem{conj}[thm]{Conjecture}
\newtheorem{exmp}[thm]{Example}

\theoremstyle{remark} 
\newtheorem*{rem}{Remark} 
\newtheorem*{note}{Note} 
\newtheorem{case}{Case} 

\newcommand{\expc}[1]{\mathbb{E}\left[#1\right]}

\newcommand{\Q}{\mathbb{Q}}
\newcommand{\R}{\mathbb{R}}
\newcommand{\C}{\mathbb{C}}
\newcommand{\Z}{\mathbb{Z}}
\newcommand{\F}{\mathbb{F}}
\newcommand{\Ga}{\mathbb{G}}

\newcommand{\nth}{n\textsuperscript{th}}
\newcommand{\bigo}[1]{\mathcal{O}(#1)}

% \setlength{\topmargin}{0.2cm}
% \setlength{\footskip}{0.2cm}
% \setlength{\hoffset}{-1cm}
% \setlength{\voffset}{-2cm}

        
\usepackage{hyperref}
        
\title{Category Theory}                               % Document Title
\author{Edward O'Catlaghan}
%\date{}                                           % Activate to display a given date or no date



\begin{document}
\maketitle \tableofcontents \clearpage


\section{Introduction} % (fold)
\label{sec:introduction}
In category theory, a branch of mathematics, a natural transformation provides a way of transforming one functor into another while respecting the internal structure (i.e. the composition of morphisms) of the categories involved. Hence, a natural transformation can be considered to be a ``morphism of functors''. Indeed this intuition can be formalized to define so-called functor categories. Natural transformations are, after categories and functors, one of the most basic notions of category theory and consequently appear in the majority of its applications.

\begin{defn}[Natural Transformation]
	Let $F$ and $G$ be functors between the categories $\mathcal{C}$ and $\mathcal{K}$, then a $\emph{natural transformation}$ $\eta : F \to G$
	associates to every object $X \in \mathcal{C}$ a morphism $\eta_{X} : F(X) \to G(X)$ between objects of $\mathcal{K}$, called the $\emph{component}$
	of $\eta$ at $X$, such that for every morphism $f : X \to Y \in \mathcal{C}$ we have:
	\\
	$\eta_{Y} \circ F(f) = G(g) \eta_{X}$
\end{defn}
Abstracting yet again, constructions are often "naturally related" – a vague notion, at first sight. This leads to the clarifying concept of natural transformation, a way to "map" one functor to another. Many important constructions in mathematics can be studied in this context. "Naturality" is a principle, like general covariance in physics, that cuts deeper than is initially apparent.

\end{document}% End of document.
