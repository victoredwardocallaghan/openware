% Copyright © 2012 Edward O'Callaghan. All Rights Reserved.

\section{Sigma Fields}
\label{sec:sigma-fields}

\begin{defn}[Sigma Field]
	A $\sigma$-field, or $\sigma$-algebra, is some
	subset $\mathcal{A} \subseteq 2^{X}$ of the power set
	$2^{X}$ of set $X$ such that;
	\begin{itemize}
		\item $\emptyset, X \in \mathcal{A}$,
		\item if $A \in \mathcal{A}$ then $A^c \in \mathcal{A}$,
		\item for any countable collection of sets
			$A_1, A_2, \dots \in \mathcal{A}$ we have,
			$\displaystyle \bigcap_{i} A_i \in \mathcal{A}$ and
			$\displaystyle \bigcup_{i} A_i \in \mathcal{A}$.
	\end{itemize}
\end{defn}

\begin{lem}
	The $\sigma$-field $\mathcal{F}$ from some set $\Omega$, taken as
	$\mathcal{F} = \{ \emptyset, \Omega \}$, is the
	$\emph{trivial}$ $\sigma$-field.
\end{lem}

\begin{defn}[Measure]
	A function $\mu$ defined on some $\sigma$-field $\mathcal{A}$
	is said to be countably additive and nonnegative $\emph{measurable}$
	provided;
	\begin{itemize}
		\item $\mu (\emptyset) = 0$,
		\item $0 \leq \mu (A) \leq \infty$ for any $A \in \mathcal{A}$,
		\item for a countable collection of sets
			$A_1, A_2, \dots \in \mathcal{A}$ with
			$A_i \cap A_j = \emptyset : i \neq j$, we have
			$\displaystyle \mu \left( \bigcup_i A_i \right) = \sum_i \mu (A_i)$.
	\end{itemize}
\end{defn}

\begin{defn}[Measure Space]
	The tuple $(X, \mathcal{A})$ is called a $\emph{measurable space}$
	for some $\sigma$-field define on some set $X$.
	That is, a space in which we may define some $\emph{measure map}$
	$\mu: \mathcal{A} \to \R$.
\end{defn}

\begin{thm}
	Let $\mu$ be a measure on $(\Omega, \mathcal{F})$
	\begin{enumerate}[i.)]
		\item \textbf{Monotonicity}: If $A \subset B$ then $\mu(A) \leq \mu(B)$.
		\item \textbf{Subadditivity}: If $A \subset \cup_{k=1}^{\infty} A_k$ then
			$\mu(A) \leq \sum_{k=1}^{\infty} \mu(A_k)$.
		\item \textbf{Continuity from below}: If $A_i \uparrow A$
			(i.e., $A_1 \subset A_2 \subset \dots$ and $\cup_i A_i = A$) then
			$\mu(A_i) \uparrow \mu(A)$.
		\item \textbf{Continuity from above}: If $A_i \downarrow A$
			(i.e., $A_1 \supset A_2 \supset \dots$ and $\cap_i A_i = A$),
			with $\mu(A_1) \leq \infty$ then $\mu(A_i) \downarrow \mu(A)$.
	\end{enumerate}
\end{thm}

\begin{proof}
	$\newline$
	\begin{enumerate}[i.)]
		\item Let $B-A = B \cap A^{c}$ be the $\emph{difference}$ of the two sets.
			Using $\oplus$ to denote disjoint unions, $B=A\oplus(B-A)$ so
			\[
				\mu(B) = \mu(A) \oplus \mu(B-A) \geq \mu(A).
			\]
		\item Let $A'_n = A_n \cap A, B_1 = A'_1$ and for $n>1$,
			$B_n = A'_n - \cup_{k=1}^{n-1}(A'_k)^{c}$. Since the $B_n$ are disjoint
			and have union $A$, we have, using the definition of measure,
			$B_k \subset A_k$, and by $i.)$ of this theorem,
			\[
				\mu(A) = \sum_{k=1}^{\infty} \mu(B_k)
				\leq \sum_{k=1}^{\infty} \mu(A_k).
			\]
		\item Let $B_n = A_n - A_{n-1}$. Then the $B_n$ are disjoint and
			have $\cup_{k=1}^{\infty} B_k = A$, $\cup_{k=1}^{n} B_k = A_n$ so
			\[
				\mu(A) = \sum_{k=1}^{\infty} \mu(B_k)
				= \lim_{n \to \infty} \sum_{k=1}^{n} \mu(B_k)
				= \lim_{n \to \infty} \mu(A_n).
			\]
		\item Given that $A_1 - A_n \uparrow A_1 - A$ so $iii.)$ implies
			$\mu(A_1 = A_n) \uparrow \mu(A_1 - A)$. Now since $A_1 \supset B$,
			we have $\mu(A_1 - B) = \mu(A) - \mu(B)$ and it follows that
			$\mu(A_n) \downarrow \mu(A)$.
	\end{enumerate}
\end{proof}

\begin{defn}[Probability Measure]
	A $\emph{probability measure}$ is some measure $\mathbb{P}$ defined
	on some $\sigma$-field $\mathcal{F}$ from some set $\Omega$ such that
	$\mathbb{P} (\Omega) = 1$.
	In particular, we call the set $\Omega$ the $\emph{sample space}$
	and the triple $(\Omega, \mathcal{F}, \mathbb{P})$ the
	$\emph{probability}$ measure space.
\end{defn}

In a $\emph{probability space}$ $(\Omega, \mathcal{F}, \mathbb{P})$
we have the $\emph{set of outcomes}$ $\Omega$, or $\emph{sample space}$,
the $\emph{set of events}$ $\mathcal{F}$ and the probability measure
$\mathbb{P}:\mathcal{F} \to [0,1]$ that assigns probabilities to events.
Recall that $\Omega, \emptyset \in \mathcal{F}$ is always trivially so,
hence the probability measure $\mathbb{P}$ is defined on all outcomes
in the sample space $\Omega$.

\begin{exmp}[Discrete probability space]
	Let $\Omega=$``a countable set'', that is, finite or countably infinite.
	Take the $\sigma$-field $\mathcal{F}=2^{\Omega}$ as the power set. Consider,
	\[
		\mathbb{P}(A) = \sum_{\omega \in A} p(\omega) : p(\omega) \geq 0
	\]
	and note
	\[
		\sum_{\omega \in \Omega} p(\omega) = 1.
	\]
	Hence, for a finite set $\Omega$, we may define a $\emph{uniform}$
	probability measure $p(\omega) = \frac{1}{|\Omega|}$. Such as for
	a fair six sided dice, where $\Omega=\{1,2,3,4,5,6\}$ and $p(\omega)=1/6$.
	In particular, for some $\emph{event}$ $A \in \mathcal{F}$ where
	$A=$``we get a 2 and a 3'', we have chance $\mathbb{P}(A)=1/6 + 1/6 = 1/3$.
\end{exmp}

\begin{thm}
	Suppose we are given some collection of $\sigma$-fields $\mathcal{F}_i, i \in I$,
	where the index set $I \neq \emptyset$ is arbitrary (i.e., possibly uncountable).
	Then $\cap_{i \in I} \mathcal{F}_i$ is a $\sigma$-field.
\end{thm}

\begin{proof}
	$\newline$
	\begin{enumerate}[i.)]
		\item Fix any $A \in \mathcal{F}_i$ for some $i \in I$.
			Then $A^{c} \in \mathcal{F}_i$ and so,
			if $A \in \cap_{i \in I} \mathcal{F}_i$ then
			$A^{c} \in \cap_{i \in I} \mathcal{F}_i$.
		\item Take any countable sequence of sets $\{A_k\}_{k=1}^{n}$ then
			$\cup_{k=1}^{n} A_k \in \mathcal{F}_i$ and so,
			if $\{A_k\}_{k=1}^{n} \in \cap_{i \in I} \mathcal{F}_i$ then,
			$\cup_{k=1}^{n} A_k \in \cap_{i \in I} \mathcal{F}_i$.
	\end{enumerate}
\end{proof}
