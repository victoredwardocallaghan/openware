% Copyright © 2012 Edward O'Callaghan. All Rights Reserved.

\section{The Console} % (fold)
\label{sec:console}

The \emph{console} is essentially the way in which we interact
with a computer. The console is just a fancy name for the mouse,
keyboard and screen set. Before graphical enabled console was
introduced consoles were presented as a terminal. Today our
consoles are graphical and so we have \textbf{terminal emulators}.

In Microsoft Windows the emphasis is on the windowing graphical
side of the console whereas GNU/Linux emphasises \emph{the terminal}
as the preferred choice of control. This is not to say that
GNU/Linux is not capable of a comprehensive array of graphical
capabilities, in fact the case is quite the opposite. However,
the terminal provides greater flexibility and more precise control
of the system console and most importantly, efficiently, the
primary reason for desktop computing in many respects. Some
drawbacks exist as with any trade off, with increased functionality
there is increased complexity. The GNU/Linux paradigm of simplicity
leads to vast flexibility and so inherently can be a little tough
to understand at first sight coming from a graphical only background.

To motivate consider the summary that, if there is no button to click
then you can't do it, whereas in the terminal when there is a will
there is a way! Thus, some old ideas should be carefully considering
before throwing them away so readily. Perhaps for some wizy-bang pretty
looking graphical interface some company is trying to sell you but really
does not actually add to your productivity in reality. 

Another main task of the terminal emulator is to authenticate the user
upon login, although this is typically done by the \emph{virtual terminal}.
The virtual terminal is the terminal emulator that is run by the kernel
itself.

Various terminal emulators exist however we consider \emph{urxvt} as
the best blend of simplicity and stability while maintaining various
useful features. The \emph{command interpreter} we shall use runs inside
the terminal emulator and allows the machine to interpret user command input.

Various command interpreters exist (i.e., zsh, ksh93, bash,...),
again giving you the freedom of choice. However \textbf{bash} or
Born Again SHell is the typically most commonly found command interpreter,
or just \emph{shell}. Since the shell is an interpreter each shell has
its own little language and so you can \emph{script} how you wish the
console to play. That is, you write \textbf{shell scripts} to automate
common tasks in our shell that you would normally do at your console.
For example, rename a directory of pictures according to some naming scheme
such as by date-location. For such a task, as a graphical user, you would
typically download some random special program in the hope that it would
do something similar to what you wish or you may spend hours manually
renaming each file at a time. This is where the power of interacting with
the console in a command fashion comes into its own. Here, we need only
specify the task in terms of a chain of a few short commands in some file,
which we call a script, and then run this file in the shell interpreter.
The shell interpreter will interpret the script as if you were siting in
front of the console doing the task and automate this task exactly how
you specified. The detail of interacting in this way with the console
will take some time to get use to and only comes with practice however
give yourself time and take it easy.
