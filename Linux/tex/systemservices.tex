% Copyright © 2012 Edward O'Callaghan. All Rights Reserved.

\section{System Services} % (fold)
\label{sec:systemservices}

System services are the client applications that work in the
background when the system boots. These services provide
various house keeping and user services which are called
demons. Some of these demons you interact with directly such
as SLiM and others you most likely don't even think about
such as NTP.

\begin{exmp}[Various common services]
	Some example services are as follows;
	\begin{itemize}
		\item NTP - Keeps the system clock in sync with the time
			from the internet.
		\item SLiM - Provides a graphical login authentication
			service for the window manager though the X server.
		\item Autofs - Checks for things like USB storage devices
			and automatically mounts them for you.
		\item WPA\_Supplicant - Configures wireless connections
			by automatically authenticating with the access point
			and connecting for you.
		\item DHCPD - Automatically sets up your IP address and
			DNS settings for your network connection.
	\end{itemize}
\end{exmp}

Various GNU/Linux distributions come with their own system
service manager. However, a typically common one is called
\emph{systemd}. SystemD handles when services should start/stop
and reports on their status should you wish to inquire about it.
Consult the man page for \emph{systemd} for details.
