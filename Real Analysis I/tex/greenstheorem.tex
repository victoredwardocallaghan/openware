% Copyright © 2012 Edward O'Callaghan. All Rights Reserved.
% !Tex root = Real Analysis I.tex

\section{Green's theorem} % (fold)
\label{sec:greenstheorem}

Green's theorem is a two dimensional analog of the
Fundamental Theorem of Calculus.

\begin{thm}
	Let $\mathbf{F}(x,y) = \left( F(x,y), G(x,y) \right)$ be,
	$\mathcal{C}^1(\bar{\Omega})$, a continuous vector field on an
	open set containing domain $\Omega \subseteq \R^2$ whose boundary
	curve, $C = \partial \Omega$, is closed and piecewise smooth.
	Then, by considering $\partial \Omega$ to have positive,
	or counterclockwise, orientation, we have:
	\[
		\iint_{\Omega} \left(
		\frac{\partial G(x,y)}{\partial x} - \frac{\partial F(x,y)}{\partial y}
		\right) \mathrm{d}y \, \mathrm{d}x =
		\int_{\partial \Omega} \left( F(x,y) \mathrm{d}x + G(x,y) \mathrm{d}y \right).
	\]
\end{thm}

\begin{proof}
	\begin{align*}
		\intertext{First suppose $\Omega$ is an elementary region of the form:}
		\Omega = \{ (x,y) : a \leq x \leq b &, \phi_1(x) \leq y \leq \psi_1(x) \}
		\tag{$x$-simple}
		\intertext{where $\phi_1,\psi_1 \in \mathcal{C}([a,b])$ and}
		\Omega = \{ (x,y) : c \leq y \leq d &, \phi_2(y) \leq x \leq \psi_2(y) \}
		\tag{$y$-simple}
		\intertext{where $\phi_2,\psi_2 \in \mathcal{C}([c,d])$.}
		\intertext{Since $\mathbf{F}(x,y) \in \mathcal{C}^1(\bar{\Omega})$ then
		$F(x,y) \in \mathcal{C}^1(\bar{\Omega})$. Hence we first show,}
		- \iint_{\Omega} \frac{\partial F}{\partial y} \mathrm{d}y \, \mathrm{d}x
		&= \int_{\partial \Omega} F(x,y) \mathrm{d}x.
		\intertext{By writing $\Omega$ as $x$-simple we see that,}
		\int_{\partial \Omega} F(x,y) \mathrm{d}x
		= \int_{C_1} F \mathrm{d}x + \underbrace{\int_{C_2} F \mathrm{d}x}_{\dag}
		& + \int_{C_3} F \mathrm{d}x + \underbrace{\int_{C_4} F \mathrm{d}x}_{\dag}.
		\intertext{Observe$^\dag$ that the curves $C_2,C_4$ are the vertical line
			portions $x=a$ and $x=b$ respectively. So any parametrisation $x'(t)=0$
			of constant terms gives us $\mathrm{d}x=0$ and so the sums are zero measure.}
			\intertext{Consider also the parameterisations $\gamma_1(x)=(x,\phi_1(x))$ and
				$\gamma_3(x)=(x,\psi_1(x))$ for the curves $C_1$ and $C_3$ respectively with
			$x \in [a,b]$. Since $C_3$ has negative orientation, we have}
			\int_{\partial \Omega} F(x,y) \mathrm{d}x
			&= \int_{a}^{b} \left( F(x,\phi_1(x)) - F(x,\psi_1(x)) \right) \mathrm{d}x.
			\intertext{Holding $x$ fixed we have, by the Fundamental Theorem of Calculus, the following}
			\iint_{\Omega} \frac{\partial F(x,y)}{\partial y} \mathrm{d}y \, \mathrm{d}x
			&= \int_{a}^{b} \left( \int_{\phi_1(x)}^{\psi_1(x)}
			\frac{\partial F(x,y)}{\partial y} \mathrm{d}y \right) \mathrm{d}x
			\intertext{and so,}
			\iint_{\Omega} \frac{\partial F(x,y)}{\partial y} \mathrm{d}y \, \mathrm{d}x
			&= - \int_{\partial \Omega} F(x,y) \mathrm{d}x. \tag{1}
			\intertext{Similarly, expressing $\Omega$ as a $y$-simple region, we obtain,}
			\iint_{\Omega} \frac{\partial G(x,y)}{\partial x} \mathrm{d}y \, \mathrm{d}x
			&= \int_{\partial \Omega} G(x,y) \mathrm{d}y. \tag{2}
			\intertext{By adding (1) and (2) together we have the required result in this simple case.}
	\end{align*}
\end{proof}
