% Copyright © 2012-2013 Edward O'Callaghan. All Rights Reserved.
% !Tex root = Real Analysis I.tex

\section{Metric Spaces} % (fold)
\label{sec:metricspaces}

\begin{defn}[Metric]
  A \emph{metric} $d_X$ on set $X$ is a function
  $d_X : X \times X \to \R$ such that, for any $x,y,z \in X$;
  \begin{enumerate}[i.)]
      \item $d_X (x,y) \geq 0$ and $d_X (x,y)=0$ iff $x=y$ (semi-positive definate),
      \item $d_X (x,y)=d_X (y,x)$ (symmetric),
      \item $d_X (x,z) \leq d_X (x,y) + d_X (y,z)$ (triangle inequality).
  \end{enumerate}
\end{defn}

\begin{defn}[Metric Space]
  A \emph{metric space} is the pair $(X, d_X)$ where $X$ is a set
  and $d_X$ is the metric defined on the set $X$.
\end{defn}

% TODO examples of metric spaces..

\begin{defn}[Open Ball]
  An \emph{open ball} in a metric space $(X,d_X)$ with center
  $x_0 \in X$ and $\epsilon$-neighborhood with $\epsilon > 0$ is defined as the set;
  \begin{align*}
    \mathcal{B}_{\epsilon}(x_0) & \doteq
    \{ x \in X : d_X(x_0,x) < \epsilon \}.
  \end{align*}
\end{defn}

\begin{defn}[Closed Ball]
  A \emph{closed ball} in $X$ with center $x_0 \in X$ and
  $\epsilon$-neighborhood with $\epsilon > 0$ is defined as the set;
  \begin{align*}
    \overline{\mathcal{B}_{\epsilon}(x_0)} & \doteq
    \{ x \in X : d_X(x_0,x) \leq \epsilon \}.
  \end{align*}
\end{defn}

\begin{defn}[Open set]
  Let $\Omega \subseteq X$ of a metric space $(X,d_X)$. The set $\Omega$ is called
  an \emph{open set} in $X$ if; for each $x \in \Omega$ there exists
  some $\delta>0$ such that $\mathcal{B}_{\delta}(x) \subseteq \Omega$.
\end{defn}

\begin{lem}
  The whole space $X$ and the empty set $\emptyset$ are trivially open.
\end{lem}

\begin{defn}[Closed set]
  A \emph{closed set} is the complement, denoted $\Omega^c$, of
  the open set $\Omega$.
\end{defn}

\begin{lem}
  The whole space $X$ and the empty set $\emptyset$ are trivially closed.
\end{lem}

\begin{defn}[Clopen set]
  A set that is both open and closed is said to be an \emph{clopen set}.
\end{defn}

\begin{defn}[Boundary]
  The \emph{boundary} of a set $\Omega \subseteq X$ of a metric space $(X,d_X)$,
  denoted by $\delta \Omega$, is defined as;
  \begin{align*}
    \delta \Omega & \doteq \{
  x \in X : \mathcal{B}_{\epsilon}(x) \cap \Omega \neq \emptyset ,
  \mathcal{B}_{\epsilon}(x) \cap \Omega^c \neq \emptyset \}.
  \end{align*}
\end{defn}

\begin{lem}
  We have that, $\delta \Omega = \delta (\Omega^c)$, is trivially so.
\end{lem}

\begin{prop}
  Any open set does not contain any of its boundary points.
\end{prop}

\begin{proof}
  Let $\Omega \subseteq X$ be open so that $\Omega^c$ is closed. It remains that
  $\delta \Omega^c \subseteq \Omega^c$. \qedhere
\end{proof}

\begin{prop}
  A set $S$ is \emph{closed} if and only if it contains
  all its boundary $\delta S \subseteq S$.
\end{prop}

\begin{proof}
  Suppose that $S$ is closed so that $S^c$ is open.
  It follows that $S^c \cap \delta S = \emptyset$ and hence $\delta S \subseteq S$.
  Conversely, now consider some $x \in S^c$. Since $\delta S \subseteq S$,
  it follows that $x \notin \delta S$. It remains by definition of
  boundary point, we have some $\epsilon>0$ such that
  $\mathcal{B}_{\epsilon}(x) \cap S = \emptyset$.
  Thus, $\mathcal{B}_{\epsilon}(x) \subseteq S$ and so $S^c$ is open. \qedhere
\end{proof}

\begin{defn}[Interior (Point)]
  Let $\Omega \subseteq X$ be some region of a metric space $(X,d_X)$.
  A point $x \in \Omega$ is said to be an \emph{interior point} if
  there exists some $\delta>0$ such that $\mathcal{B}_{\delta}(x) \subseteq \Omega$.
  The set of all interior points, denoted $\Omega^{\circ}$,
  is called the \emph{interior} of the region $\Omega$.
\end{defn}

\begin{defn}[Closure]
  Let $S \subseteq X$. We define the \emph{closure} of $S$, denoted
  $\overline{S}$, by the set $\overline{S}=S \cup \delta S$.
\end{defn}

\begin{defn}[Accumulation Point]
  Let $\Omega \subseteq X$ and fix some point $x \in X$. We call $x$
  an \emph{accumulation}, or \emph{limit}, \emph{point} of
  $\Omega$ if every open ball around $x$ contains atleast one distinct point
  $y \in \Omega$. In particular, for every $\epsilon > 0$ we have that,
  $(\mathcal{B}_{\epsilon}(x) - {y}) \cap \Omega \neq \emptyset$.
\end{defn}

\section{Convergence} % (fold)
\label{sec:convergence}

\begin{defn}[Limit]
	Let $\{x_k\}_{k=1}^{\infty}$ be a sequence of
	vectors in $\R^n$ and $x \in \R^n$. We say that
	$\{x_k\}_{k=1}^{\infty}$ $\emph{converges}$ to
	a $\emph{limit}$ $x$ if,
	\[
		\lim_{k \to \infty} d(x_k,x) =
		\lim_{k \to \infty} \| x_k - x \| \to 0 \in \R
	\]
	written $x_k \to x$.
\end{defn}

\section{Compactness} % (fold)
\label{sec:compactness}

Let $I$ denote any, possibly infinite, indexing set.

\begin{defn}[Open Cover]
	A $\textbf{open cover}$ of a set $S \subseteq \R^n$
	is a collection $\{V_i\}_{i \in I}$ of open sets of
	$\R^n$ such that $S \subset \bigcup_{i \in I} V_i$.
	A $\emph{subcover}$ is a subcollection which also
	covers $S$.
\end{defn}

\begin{defn}[Compact]
	A subset $S \subseteq \R^n$ is said to be $\textbf{compact}$
	if from every open cover of $S$ we may find a $\emph{finite}$
	subcover of $S$.
\end{defn}
