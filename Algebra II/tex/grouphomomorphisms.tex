% Copyright © 2013 Edward O'Callaghan. All Rights Reserved.

\subsection{Group Homomorphisms} % (fold)
\label{subsec:homomorphisms}

Homomorphisms are structure preserving mappings. In group homomorphisms we preserve the
group structure, defined by the binary law of composition. In particular,

\begin{defn}[Group Homomorphism]
	Let $(\mathcal{G}, \circ)$ and $(\mathcal{H},\dagger)$ be two groups. Then a mapping
	$\varphi : \mathcal{G} \to \mathcal{H}$ is called a $\emph{group homomorphism}$ if
	\[
		\varphi(g_1 \circ g_2) = \varphi(g_1) \dagger \varphi(g_2) : g_1,g_2 \in \mathcal{G}.
	\]
\end{defn}

It follows that, for some $g \in \mathcal{G}$ we have,
\begin{align*}
	\varphi ( e_g ) &= \varphi (g \circ g^{-1})
	\\
	&= \varphi (g) \dagger \varphi (g^{-1})
	\\
	&= \varphi (g) \dagger (\varphi (g) )^{-1}
	\\
	&= e_h \in \mathcal{H}.
\end{align*}
That is the identity $e$ has been preserved.

In this way, it does not matter if we compose in $\mathcal{G}$ and map to $\mathcal{H}$ or take two elements
in $\mathcal{G}$ then compose the mapped elements in $\mathcal{H}$, since the group structure has been preserved.

How much information about the elements inside the structure is, however, another quality to consider. Hence we fix some
terminology here.
\begin{itemize}
	\item A homomorphism that is injective is called monomorphic.
	\item A homomorphism that is surjective is called epimorphic.
	\item A homomorphism that is bijective is called isomorphic.
\end{itemize}
Thus we have the following definitions by considering a group homomorphism $\varphi : \mathcal{G} \to \mathcal{H}$.

\begin{defn}[Monomorphic]
	$\varphi$ is $\textbf{monomorphic}$ if for $\varphi(x) = \varphi(y) \implies x = y \, \forall x,y \in \mathcal{G}$.
\end{defn}

\begin{defn}[Epimorphic]
	$\varphi$ is $\textbf{epimorphic}$ if $\forall h \in \mathcal{H} \exists g \in \mathcal{G}$ so that $\varphi(g) = h$.
\end{defn}

\begin{defn}[Isomorphic]
	$\varphi$ is $\textbf{isomorphic}$ if $\varphi$ is $\textbf{both}$ mono- and epic- morphic.
\end{defn}

Some special cases are sometimes of particular interest and we shall outline them now.

\begin{defn}[Endomorphic]
	A monomorphism $\mathcal{G} \to \mathcal{G}$ for a group $\mathcal{G}$
	is called an $\emph{endomorphism}$ of $\mathcal{G}$.
\end{defn}

\begin{defn}[Automorphic]
	A isomorphism $\mathcal{G} \to \mathcal{G}$ for a group $\mathcal{G}$
	is called an $\emph{automorphism}$ of $\mathcal{G}$.
\end{defn}

\begin{rem}
	The set $Aut(\mathcal{G})$ of automorphisms of $\mathcal{G}$ forms a group, when composition of
	mappings is taken as the group law of composition.
\end{rem}

\begin{exmp}[Trivial Homomorphism]
	The trivial group homomorphism $id_{\mathcal{G}} : \mathcal{G} \to \mathcal{G}$, given
	by the mapping $g \mapsto g$ for every $g \in \mathcal{G}$, is in fact a group automorphism.
\end{exmp}

\begin{exmp}
	Consider $\psi : GL_n(\R) \to \R^{\times}$ defined by the mapping
	$A \mapsto \det(A)$ and recall that $\det(AB) = \det(A) \det(B)$.
	That is, the determinant is a group homomorphism.
\end{exmp}

\begin{exmp}
	Consider $\psi : \mathcal{G} \to S_n / A_n$ where $\mathcal{G}=\{-1,1\}$, defined by
	$1 \mapsto A_n$ and $-1 \mapsto (1 \, 2) A_n$, and observe that $\phi$ is a group homomorphism.
\end{exmp}

\begin{prob}
	Consider the map $\phi : \R \to SL_2(\R)$ defined by,
	\[
		x \mapsto
		\begin{pmatrix}
			1 & x \\
			0 & 1
		\end{pmatrix}.
	\]
	Show that $\phi(x + y) = \phi(x) \cdot \phi(y)$.
	Also, prove that $\phi$ is injective.
\end{prob}

\begin{exmp}
	Consider the map $\exp : \R^{+} \to \R^{\times}$ from the additive to the
	multiplicative group, defined by $x \mapsto e^x$, is a group homomorphism.
	Since, $\exp{(x + y)} = \exp{(x)} \cdot \exp{(y)}$.
\end{exmp}

\begin{exmp}
	Consider the linear transformation $T : \mathcal{V} \to \mathcal{W}$.
	By definition of linearity, $T(\vec{v}_1 + \vec{v}_2) = T(\vec{v}_1) + T(\vec{v}_2)$,
	the mapping $T$ is a group homomorphism from the additive group of vector space $\mathcal{V}$
	to the additive group of vector space $\mathcal{W}$.
\end{exmp}

\begin{prob}
	Suppose $N \unlhd \mathcal{G}$ and $\pi : \mathcal{G} \to \mathcal{G} / N$,
	given by the mapping $g \mapsto g N$ for every $g \in \mathcal{G}$. Show that $\pi$
	is a group homomorphism and then show that it is surjective.
\end{prob}

\begin{prob}
	Suppose $\phi : \C^{\times} \to \R^{\times}$ given by the mapping $z \mapsto | z |$. Show
	that $\phi$ is a group homomorphism. Is $\phi$ bijective?
\end{prob}

\begin{prop}
	Let $\varphi : \mathcal{G} \to \mathcal{H}$ be a group homomorphism.
	\begin{enumerate}[i.)]
		\item $\varphi(1_{\mathcal{G}}) = 1_{\mathcal{H}}$,
		\item $\varphi(g^{-1}) = \varphi(g)^{-1}$ for all $g \in \mathcal{G}$,
		\item If $\mathcal{G}' \leq \mathcal{G}$ then $\varphi(\mathcal{G}') \leq \mathcal{H}$ when the restriction
			$\mathcal{H} = \varphi |_{\mathcal{G}'} (\mathcal{G})$ holds,
		\item If $\varphi$ is an isomorphism, then so is its inverse $\varphi^{-1} : \mathcal{H} \to \mathcal{G}$,
		\item If $\psi : \mathcal{G} \to \mathcal{H}$ and $\phi : \mathcal{H} \to \mathcal{K}$ are group
			homomorphisms then so is $\phi \circ \psi$.
	\end{enumerate}
\end{prop}

\begin{proof}
	For $i.)$ we see that,
	\begin{align*}
		1_{\mathcal{H}} \cdot \varphi(1_{\mathcal{G}}) &= \varphi(1_{\mathcal{G}}) \tag{and that} \\
		\varphi(1_{\mathcal{G}}) &= \varphi(1_{\mathcal{G}} \circ 1_{\mathcal{G}}) \\
		&= \varphi(1_{\mathcal{G}}) \cdot \varphi(1_{\mathcal{G}})
		\intertext{so we have that,}
		1_{\mathcal{H}} \cdot \varphi(1_{\mathcal{G}}) &= \varphi(1_{\mathcal{G}}) \\
		\Rightarrow 1_{\mathcal{H}} \cdot \varphi(1_{\mathcal{G}}) \cdot \varphi(1_{\mathcal{G}})^{-1}
		&= \varphi(1_{\mathcal{G}}) \cdot \varphi(1_{\mathcal{G}})^{-1} \\
		\Rightarrow 1_{\mathcal{H}} &= \varphi(1_{\mathcal{G}}). \qedhere
	\end{align*}
\end{proof}

\begin{proof}
	For $ii.)$ we see that,
	\begin{align*}
		g g^{-1} = 1_{\mathcal{G}} &= g^{-1} g \\
		\Rightarrow 1_{\mathcal{H}} &= \varphi(g) \varphi(g^{-1}) \\
		&= \varphi(g^{-1}) \varphi(g).
		\intertext{Hence,}
		\varphi(g^{-1}) &= \varphi(g)^{-1}. \qedhere
	\end{align*}
\end{proof}

\begin{proof}
	For $iii.)$ we check for closure and inverses as follows.
	Define:
	\begin{align*}
		\varphi(\mathcal{G}') & \doteq \{ \varphi(g) : g \in \mathcal{G}' \} \\
		\varphi(g) \varphi(g') &= \varphi(g g') \in \varphi(\mathcal{G}') : g,g' \in \mathcal{G}'
		\intertext{and so $\varphi(\mathcal{G}')$ is closed. Now given,}
		\varphi(g) & \in \varphi(\mathcal{G}') : g \in \mathcal{G}'
		\intertext{then we have that,}
		\varphi(g)^{-1} = \varphi(g^{-1}) & \in \varphi(\mathcal{G}') : g^{-1} \in \mathcal{G}'
		\intertext{and so $\varphi(\mathcal{G}')$ has inverses. Hence $\varphi(\mathcal{G}')$ is a subgroup.}
	\end{align*}
\end{proof}

\begin{proof}
	For $iv.)$ we consider the mapping $\varphi^{-1} : \mathcal{H} \to \mathcal{G}$ defined by
	$h \mapsto g$ if $\varphi(g)=h$. First we check that $\varphi^{-1}$ is a group homomorphism, that is:
	\begin{align*}
		\varphi(h h')^{-1} &= \varphi(h)^{-1} \varphi(h')^{-1}.
		\intertext{Suppose that $g=\varphi(h)^{-1}$ and $g'=\varphi(h')^{-1}$ and since,}
		\varphi(g g') &= \varphi(g) \varphi(g') \\
		&= h h' \\
		\Rightarrow \varphi(h h')^{-1} &= g g' \\
		&= \varphi(h)^{-1} \varphi(h')^{-1}
	\end{align*}
	and so $\varphi^{-1}$ is a group homomorphism. It now trivially follows that $\varphi^{-1}$ is
	isomorphic if $\varphi$ is isomorphic. \qedhere
\end{proof}

\begin{proof}
	For $v.)$ we see directly that, given $\psi(g g') = \psi(g) \cdot \psi(g')$ and
	$\phi(h h') = \phi(h) \dagger \phi(h')$, we have that:
	\begin{align*}
		\phi \circ \psi (g g') &= \phi ( \psi(g) \cdot \psi(g') ) \\
		&= \phi \circ \psi(g) \dagger \phi \circ \psi(g') \\
		&= \phi(h) \dagger \phi(h') = \phi(h h'). \qedhere
	\end{align*}
\end{proof}

\begin{defn}[kernel]
	If $\varphi : \mathcal{G} \to \mathcal{H}$ is a group homomorphism,
	then the $\emph{kernel}$ is the set
	$\ker(\varphi) = \{ g \in \mathcal{G} : \varphi (g) = e_h \in \mathcal{H} \}$.
\end{defn}

If $\varphi : \mathcal{G} \to \mathcal{H}$ is a group homomorphism, then
observe that $\ker(\varphi)$ is a normal subgroup of of $\mathcal{G}$.
