% Copyright © 2013 Edward O'Callaghan. All Rights Reserved.

\section{Definition} % (fold)
\label{sec:definition}

The $\lcal$ is defined by the concept of an ``expression''
defined in the following way:

\begin{defn}[Expression]
	An \emph{expression} in $\lcal$ is defined recursively as:
	\begin{align*}
		<expression> & \doteq <name> | <function> | <application>
		\\
		<function> & \doteq \lambda <name>.<expression>
		\\
		<application> & \doteq <expression><expression>
	\end{align*}
	where ``name'' (or ``variable'') is some arbitrary identifier.
\end{defn}

A trivial example of a $\lambda$-expression is given here.

\begin{exmp}[Identity function]
	The following $\lambda$-expression defines the identity function:
	\[
		\lambda x.x
	\]
\end{exmp}

Arguments to functions have no relevance on the behaviour of the function
and so serve only as place holders to specify the substitution rule
when the function is evaluated. So we can write various synonyms for
the above identity function as,
\[
	(\lambda x.x) \equiv (\lambda y.y) \equiv (\lambda t.t)
\]
and so forth.

Notice that an expression \textbf{E} is usually enclosed in parenthesis
for clarity, that is (\textbf{E}). We adopt the convention that
function application \emph{associates} from the left, that is
we evaluate expressions in the following way:
\[
	\textbf{E}_1 \textbf{E}_2 \textbf{E}_3 \dots \textbf{E}_n
	= (\dots ((\textbf{E}_1 \textbf{E}_2) \textbf{E}_3) \dots \textbf{E}_n).
\]

\begin{defn}
	Their are only two \emph{keywords} used in the language,
	$\lambda$ and the \emph{dot}.
	\begin{itemize}
		\item The \emph{name} after $\lambda$ the identifier of the
			\emph{argument} of a function.
		\item The \emph{expression} after the dot is called the ``body'',
			or \emph{definition} of a function.
	\end{itemize}
\end{defn}

The application of a function to an expression is made clear by example:

\begin{exmp}
	\begin{align*}
		(\lambda x.x)y &= [y/x]x
		\\
		&= y.
	\end{align*}
	That is, we substitute $x$ by $y$ in the expression to the right, which
	is simply $x$ and so we have $y$.
\end{exmp}
