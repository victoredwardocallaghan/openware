% Copyright © 2012 Edward O'Callaghan. All Rights Reserved.

\section{Topological Space} % (fold)
\label{sec:topologicalspace}

Topological spaces allow for the formal definition of
$\emph{convergence, continuity and connectedness}$.
These are central concepts found typically in analysis.
However topology, the study of topological spaces, is more
general and allows for study of the mechanisms behind analysis
in this light.
A mathematical hierarchy of generality can be view in this way;
\begin{center}
	Normed vector spaces $\subset$ Metric spaces $\subset$ Topological spaces.
\end{center}

In truth, topology captures the main aspects of geometry however
dispenses with some notions, in particular, that of an angle. So
topology is geometry with no such notion of angle and thus more general.
Here is a light example:

\begin{exmp}
	\[
		\heartsuit = 0 = \square.
	\]
	However,
	\[
		\angle \neq \square.
	\]
\end{exmp}

\begin{defn}[Topological space]
	A $\emph{topological space}$ is a pair $(X, \mathcal{T})$
	where $X$ is a set and $\mathcal{T} \subset 2^{X}$ such that:
	\begin{itemize}
		\item $\emptyset, X \in \mathcal{T}$;
		\item For $\{U_i\}_{i=1}^{n} \subseteq \mathcal{T}$ we have 
			$\displaystyle \bigcap_{i=1}^{n} U_i \in \mathcal{T}$;
		\item For $U_{\lambda} \in \mathcal{T} : \lambda \in \Lambda$ give some
			an arbitrary indexing set $\Lambda$ we have
			$\displaystyle \bigcup_{\lambda \in \Lambda} U_{\lambda} \in \mathcal{T}$.
	\end{itemize}
	In particular $\mathcal{T}$ is \textbf{closed under} $\emph{finite}$ intersections
	and $\emph{arbitrary}$, possibly uncountably infinite, unions. We may denote
	the pair $(X, \mathcal{T})$ by $\mathcal{T}_{X}$.
\end{defn}

If $(X,\mathcal{T})$ is a topological space, we call $\mathcal{T}$
a topology on $X$. A set $U \in \mathcal{T}$ is called the
$\emph{open set}$ of the topology $\mathcal{T}$.

\begin{prob}
	Pick any arbitrary metric space $M=(X,d)$ and show this
	induces a topology $\mathcal{T}_{X}$. In particular,
	the metric topological space $(M, \mathcal{T}_{M})$.
\end{prob}

$V \subseteq X$ is $\emph{closed}$ if its complement is $\emph{open}$.
The topology could be defined equivalently by the collection of closed
sets, which enjoys finite unions and arbitrary intersection.
If $Z \subset X$, the $\emph{closure}$ of $Z$, denoted $\bar{Z}$, is
the intersection of all closed sets containing $Z$. By the arbitrary
intersection property of closed sets, $\bar{Z}$ is $\emph{closed}$.
A $\emph{neighbourhood}$ of a point $x \in X$ is any open subset $V \subset X$
containing $x$.

\begin{defn}[Discrete topology]
	A $\emph{discrete topology}$ has every subset open.
\end{defn}

\begin{defn}[Indiscrete topology]
	A $\emph{indiscrete topology}$ has no open sets except $\emptyset$ and $X$ itself.
\end{defn}

\begin{defn}[Closure operator]
	A $\emph{closure operator}$ is a function
	\[
		\operatorname{cl}: 2^{X} \to 2^{X}
	\]
	that satisfies, for $A,B \in 2^{X}$,
	\begin{itemize}
		\item $A \subset \operatorname{cl}(A)$ (extensively);
		\item $\operatorname{cl}(\operatorname{cl}(A)) = \operatorname{cl}(A)$ (idempotent);
		\item $\operatorname{cl}(\emptyset)=\emptyset$ (preserves nullary unions);
		\item $\operatorname{cl}(A \cup B) = \operatorname{cl}(A) \cup \operatorname{cl}(B)$
			(preserves binary unions).
	\end{itemize}
	Note here that $2^{X}$ denotes the $\emph{power set}$ of $X$.
\end{defn}

We may redefine a topological space equivalently in terms of a closure operator
in the following way:

\begin{defn}[Topological space - alternative]
	A $\emph{topological space}$ $(X, \operatorname{cl})$ is a set $X$
	endowed with closure operator from the power set of $X$ to itself
	$\operatorname{cl}: 2^{X} \to 2^{X}$.
\end{defn}

Notice that we may recover the topological definitions in terms of the
closure operator.
By considering a function $f: (X, \operatorname{cl}) \to (Y, \operatorname{cl}')$
between topological spaces $(X, \operatorname{cl})$ and $(Y, \operatorname{cl}')$.
See that $f$ is then said to be continuous if, for all $A \subset X$, we have
$f(\operatorname{cl}(A)) \subset \operatorname{cl}'(f(A))$. A point
$x \in A$ is closed in $(X,\operatorname{cl})$ if $x \in \operatorname{cl}(A)$
and the set $A$ is closed in $(X, \operatorname{cl})$ if $A=\operatorname{cl}(A)$.
