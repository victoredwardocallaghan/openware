\documentclass{owmaths}

\usepackage{owshortcuts}
\usepackage[all]{xy}
\usepackage{csquotes}

\begin{document}

\subject{Measure Theory}
\author{Edward McDonald}
\title{Elementary measure theory from an abstract point of view}
\studentno{616}


\setlength\parindent{0pt}

\section{Introduction}
The purpose of these notes is to go over the elementary concepts of
abstract measure theory. There will be few original ideas of proofs. Instead
of being original, the purpose of these notes is to:
\begin{enumerate}
    \item{} Cover the basic ideas of measure theory without getting bogged
    down in specific examples, like the Carath\'erodory construction, or Lebesgue
    measure
    \item{} To include all proofs, so the measure theory is presented
    as a coherent whole
    \item{} To present ideas in as abstract a manner as possible
\end{enumerate}

\section{Concepts from order theory}
As a reminder, we provide the definition of a partially ordered set.
\begin{definition}
    A \emph{partially ordered set} (a.k.a. a poset) is a pair $(P,\leq)$
    where $\leq$ is a relation, such that: 
    \begin{enumerate}
        \item{} $\leq$ is reflexive: for all $a \in P$, $a \leq a$.
        \item{} $\leq$ is antisymmetric: if $a,b \in P$ with $a\leq b$ and $b\leq a$, then $a = b$.
        \item{} $\leq$ is transitive: if $a,b,c \in P$, with $a \leq b$ and $b\leq c$, then $a\leq c$.
    \end{enumerate}
\end{definition}
We shall make extensive use of the concepts of least upper bound
and greatest lower bound. Thus we introduce a specific notation for them.
\begin{definition}
    Let $\{a_i\}_{i \in I}$ be an indexed subset of a partially ordered set $(P,\leq)$. Then
    an element $b \in P$ is set to be the least upper bound of $\{a_i\}_{i \in I}$ if,
    for any $c \in P$ such that $a_i \leq c$ for all $i \in I$, we have $b \leq c$.
    We write symbollically,
    \begin{equation*}
        b = \bigvee_{i \in I} a_i.
    \end{equation*}
    If $x,y \in P$, we write the least upper bound of $\{x,y\}$ as $x \vee y$.
    
    Similarly, the greatest lower bound of a set $\{a_i\}_{i \in I}$ is denoted
    \begin{equation*}
        \bigwedge_{i \in I} a_i.
    \end{equation*}
    
    Theese operations are called \emph{meet} and \emph{join} respectively.
\end{definition}
In a general poset, there is no reason why the meet of
join of an arbitrary subset should exist. A lattice
is a poset where we assume that greatest lower bounds
and least upper bounds of pairs of elements always exist. That is,
\begin{definition}
    A \emph{lattice} is a poset $(P,\leq)$ where for any $a,b \in P$,
    we have $a\vee b,a\wedge b \in P$.
    
    A lattice is distributive if for any $a,b,c \in P$, we have
    \begin{equation*}
        a \vee (b \wedge c) = (a\vee b)\wedge (a\vee c)
    \end{equation*}
    or
    \begin{equation*}
        a \wedge (b \vee c) = (a \wedge b) \vee (a \wedge c).
    \end{equation*}
\end{definition}
\begin{proposition}
    One notion of distributivity implies the other.
\end{proposition}
\begin{proof}
    Suppose that for all $a,b,c \in P$, we have
    \begin{equation*}
        a \vee (b \wedge c) = (a \vee b)\wedge (a \vee c).
    \end{equation*}
    
    Then compute,
    \begin{align*}
        (a \wedge b) \vee (a \wedge c) &= (a\vee(a\wedge c)) \wedge (b\vee (a\wedge c))\\
        &= a \wedge (b \vee a) \wedge (b\vee c)\\
        &= a \wedge (b \vee c).        
    \end{align*}
    Here we have used the associative laws and absorption law, that $a \wedge (b\vee a) = a$.
    These are easy to prove.
\end{proof}



All posets that we deal with in these notes are bounded. So we define a bounded
poset.
\begin{definition}
    A poset $(P,\leq)$ is bounded if there exist elements $0,1 \in P$
    such that $0\leq a \leq 1$ for all $a \in P$.
\end{definition}

So far we have defined posets as abstract sets with relations. However, we shall
mostly be interested in \emph{concrete posets}. That is, given a set
$X$, any subset $A \subseteq \mathcal{P}(X)$ is a poset,
with the partial order being given by inclusion.

In order to introduce Boolean algebras, we need to introduce complements.
This is a concept that makes sense for bounded posets.
\begin{definition}
    Suppose that $(P,\leq,0,1)$ is a bounded poset. Then given $a \in P$,
    an element $b \in P$ is called a complement of $b$ if
    \begin{align*}
        a \wedge b &= 0\\
        a \vee b &= 1
    \end{align*}
\end{definition}
\begin{proposition}
    Complements in bounded distributive lattices are unique, if they exist.
\end{proposition}
\begin{proof}
    Let $(P,\leq,0,1)$ be a bounded distributive lattice. Suppose that $a \in P$,
    and $b,c \in P$ are complements of $a$. Then,
    \begin{equation*}
        (a \vee  b) \wedge c = c
    \end{equation*}
    Hence,
    \begin{equation*}
        (a \wedge c) \vee (b \wedge c) = c
    \end{equation*}
    Thus $b \wedge c = c$, so $b \leq c$. By symmetry, $c \leq b$. So $c = b$.
\end{proof}

Now we introduce the important concept of a Boolean algebra.
\begin{definition}
    A \emph{Boolean algebra} is a bounded distributive lattice 
    such that every element has a complement. The complement of an element
    $a$ is denoted $a'$.
\end{definition}    
\begin{remark}
    The unary $'$ operation is well defined, since complements are unique
    in a distributive lattice. It is easy to see that $'$ is involutive, that is
    $a'' = a$ for any $a$. Also, $0' = 1$.
\end{remark}
\begin{proposition}
    Suppose that $B$ is a Boolean algebra, and $a,b \in B$. Then $(a\vee b)' = a'\wedge b'$
    and $(a \wedge b)' = a' \vee b'$.
\end{proposition}
\begin{proof}
    This is a simple verification. By uniquenss, we need to check that
    \begin{equation*}
        (a \wedge b) \wedge (a' \vee b') = 1
    \end{equation*}
    and that
    \begin{equation*}
        (a \wedge b) \vee (a' \vee b') = 0.
    \end{equation*}
    This follows from distributivity.
\end{proof} 

\begin{definition}
    Given a set $X$, a Boolean algebra on $X$ is a subset of $\mathcal{P}(X)$
    ordered by inclusion, with meet and join given by intersection and union,
    bounds $\emptyset$ and $X$ 
    and complement given by set complementation.
\end{definition}    


For measure theory, Boolean algebras are not enough. We must instead 
introduce the concept of a Boolean $\sigma$-algebra.
\begin{definition}
    A Boolean $\sigma$-algebra is a Boolean algebra that is closed
    under countable meets and countable joins. That is,
    a Boolean algebra
    $A$ is a Boolean $\sigma$-algebra if for any $\{a_i\}_{i=1}^\infty \subseteq A$, we have
    \begin{equation*}
        \bigwedge_{i=1}^\infty a_i,\;\bigvee_{i=1}^\infty a_i \in A.
    \end{equation*}
    
    Furthermore we ask that countable least upper bounds and greatest lower bounds
    satisfy \emph{infinite distributive laws}: for any countable subsets $\{a_i\}_{i=1}^\infty$
    and $\{a_j\}_{j=1}^\infty$, we have
    \begin{align*}
        \left(\bigvee_{i=1}^\infty a_i\right)\wedge\left(\bigvee_{j=1}^\infty b_j \right) &= \bigvee_{i,j=1}^\infty (a_i \wedge b_j)\\
        \left(\bigwedge_{i=1}^\infty a_i\right)\vee\left(\bigwedge_{j=1}^\infty b_j\right) &= \bigwedge_{i,j=1}^\infty (a_i \vee b_j)
    \end{align*}
\end{definition}
\begin{proposition}
    In a Boolean $\sigma$-algebra, De Morgan's laws generalise to countable meets and joins.
\end{proposition}
\begin{proof}
    Again, by uniqueness of complements, we only need to check that
    \begin{equation*}
        \left(\bigvee_{i=1}^\infty a_i\right)' = \bigwedge_{i=1}^\infty a_i'.
    \end{equation*}
    and similarly with joins and meets interchanged. This is a simple verification.
\end{proof}

\begin{definition}
    A $\sigma$-algebra on a set $X$ is a subset of $\mathcal{P}(X)$
    with meet and join given by intersection and union, with bounds $\emptyset$ and $X$ 
    and complement given by set complementation.
\end{definition}

$sigma$-algebras on sets are usually specified by a generating set. This is justified
by the following proposition.
\begin{proposition}
    Let $X$ be a set, and let $\mathcal{C} \subseteq \mathcal{P}(X)$. There
    is a unique smallest $\sigma$-algebra containing $\mathcal{C}$, which we
    denote $\sigma(\mathcal{C})$.
\end{proposition}
\begin{proof}
    The key here is that $\mathcal{P}(X)$ is a $\sigma$-algebra
    containing $\mathcal{C}$, and an intersection of an arbitrary family 
    of $\sigma$-algebras is again a $\sigma$-algebra. 
    
    Hence, take the intersection of all the $\sigma$-algebras containing $\mathcal{C}$.
\end{proof}


\section{Monotone class theorem}
$\sigma$-algebras on sets are the posets which appear most frequently in measure theory.
However, also of great importance are $\pi$-classes and $d$-classes, which
are connected to $\sigma$-algebras by the monotone class theorem.

\begin{definition}
    Suppose that $X$ is a set. A $\pi$-class on $X$ is a non-empty collection of subsets
    of $X$ closed under intersection.
\end{definition}

\begin{definition}
    Suppose that $X$ is a set. A $d$-class on $X$ (also called a Dynkin class, or a monotone class, but
    we will exclusively call it a $d$-class) is a collection of subsets
    of $X$ closed under increasing countable union, that is, for any countable collection
    of subsets, $\{A_i\}_{i=1}^\infty \subseteq \mathcal{P}(X)$ with $A_i \subseteq A_j$
    for $i \leq j$, we have
    \begin{equation*}
        \bigcup_{i=1}^\infty A_i
    \end{equation*}
    in the $d$-class.
    
    A $d$ class must contain $X$.
    
    A $d$-class also must be closed under relative complements. That is, if $A,B$
    are in the $d$-class, with $B \subseteq A$ then $A \setminus B$ is in the $d$-class.
    
\end{definition}

An important feature of $d$-classes is that they can be generated by sets, similar
to $\sigma$-algebras.
\begin{proposition}
    Let $X$ be a set, and let $\mathcal{C} \subseteq \mathcal{P}(X)$
    be a collection of subsets of $X$. There is a unique smallest $d$-class
    containing $\mathcal{C}$, which we denote $d(\mathcal{C})$.
\end{proposition}
\begin{proof}
    This is identitical to the case with $\sigma$-algebras. Since $\mathcal{P}(X)$
    is a $d$-class containing $\mathcal{C}$, and any intersection of a family of $d$-classes
    is again a $d$-class, simply take the intersection of all $d$-classes containing $\mathcal{C}$.
\end{proof}

Now we state and prove the important monotone class theorem:
\begin{theorem}
    Let $X$ be a set, and suppose that $\mathcal{C}$ is a $\pi$-class
    on $X$. Then $\sigma(\mathcal{C}) = d(\mathcal{C})$.
\end{theorem}
\begin{proof}
    We clearly have that $d(\mathcal{C}) \subseteq \sigma(\mathcal{C})$,
    since any $\sigma$-algebra is a $d$-class. So we must prove
    the reverse inclusion. We can accomplish this by proving that $d(\mathcal{C})$
    is a $\sigma$-algebra. 
    
    Since $X \in \mathcal{C}$, we have that $X \setminus A \in d(\mathcal{C})$
    for any $A \in d(\mathcal{C})$ since $d(\mathcal{C})$ is closed under
    relative complements. Hence $d(\mathcal{C})$ is closed under complementation.
    
    Let us show that $d(\mathcal{C})$ is closed under finite intersections. Let
    \begin{equation*}
        \mathcal{D}_1 = \{A \in d(\mathcal{C}) \;:\;A\cap C \in d(\mathcal{C}),\text{ for all }C \in \mathcal{C}\}.
    \end{equation*}
    It is easily verified that $\mathcal{D}$ is a $d$-class, and since $\mathcal{C}$ is a $\pi$-class, we
    have $\mathcal{C} \subseteq \mathcal{D}_1$, so $d(\mathcal{C}) \subseteq \mathcal{D}_1$.
    Hence, $\mathcal{D}_1 = d(\mathcal{C})$.
    
    Now define,
    \begin{equation*}
        \mathcal{D}_2 = \{A \in d(\mathcal{C})\;:\;A\cap B \in \mathcal{D}_1,\text{ for all }B \in d(\mathcal{C})\}.
    \end{equation*}
    We have $\mathcal{C} \subseteq \mathcal{D}_2$, and again it is easily
    verified that $\mathcal{D}_2$ is a $d$-class. Hence $\mathcal{D}_2 = d(\mathcal{C})$.
    
    Thus $d(\mathcal{C})$ is closed under finite intersection. Let $A,B \in d(\mathcal{C})$. 
    Then $X \setminus A,X\setminus B \in \mathcal{C}$. Hence $A \cup B = X\setminus ((X\setminus A)\cap (X\setminus B))$.
    
    So $d(\mathcal{C})$ is closed under finite unions.
    
    Given any countable subset, $\{A_i\}_{i=1}^\infty$, we can write $\bigcup_{i=1}^\infty A_i$
    as an increasing union by defining $B_i = \bigcup_{j=1}^i A_i$, and then
    $\bigcup_{i=1}^\infty A_i = \bigcup_{i=1}^\infty B_i$. Hence
    since $d(\mathcal{C})$ is closed under finite union, and countable increasing union,
    it is closed under countable union. By taking complements, it is closed
    under countable intersection. Hence $d(\mathcal{C})$ is a $\sigma$-algebra,
    so finally we have $\sigma(\mathcal{C}) = d(\mathcal{C})$.
\end{proof}

\section{Measurable functions}
First we define measurable functions on a concrete $\sigma$-algebra, and then talk
about how the concept is generalised to abstract $\sigma$-algebras.
Despite the name, one does not need a measure to define measurable functions,
and the concept is one that is purely to do with $\sigma$-algebras, and is therefore
within the topic of order theory.
\begin{definition}
    Let $(X,\mathcal{A})$ and $(Y,\mathcal{B})$ be two sets equipped
    with $\sigma$-algebras. A function $f:X\rightarrow Y$
    is measurable if, for each $B \in \mathcal{B}$, $f^{-1}(B) \in \mathcal{A}$.
\end{definition}

There is a very important lemma, which we shall use frequently:
\begin{lemma}
    Suppose that $(X,\mathcal{A})$, $(Y,\mathcal{B})$ are $\sigma$-algebras.
    Further suppose that $\mathcal{B} = \sigma(\mathcal{C})$ for some
    collection $\mathcal{C}$ of subsets of $Y$. Then $f:X\rightarrow Y$
    is measurable if and only if $f^{-1}(C) \in \mathcal{A}$ for each $C \in \mathcal{C}$.
\end{lemma}
\begin{proof}
    One direction of implication is clear, so suppose that $f^{-1}(C) \in \mathcal{A}$
    for each $C \in \mathcal{C}$. Let $\mathcal{F}$ be the set of all $A \subset Y$
    such that $f^{-1}(A) \in \mathcal{A}$. We have $\mathcal{C} \subseteq \mathcal{F}$
    by construction. However, it is easy to see that $\mathcal{F}$ is a $\sigma$-algebra.
    Hence, $\sigma(\mathcal{C}) \subseteq F$. And we are done.
\end{proof}


\begin{proposition}
    The compostion of measurable functions is measurable.
\end{proposition}
\begin{proof}
    This is obvious.
\end{proof}

The correct way to think about measurable functions is, for a measurable
function $f:X\rightarrow Y$, we have a map $f^{-1}:\mathcal{B}\rightarrow\mathcal{A}$,
and because preimages preserve unions, intersections and complements, this is an
\emph{algebra homomorphism}. We define this now.
\begin{definition}
    Let $A$ and $B$ be abstract Boolean $\sigma$-algebras. A function $f:A\rightarrow B$
    is an algebra homomorphism if, for any collection $\{a_i\}_{i=1}^\infty \subseteq A$, we have
    \begin{align*}
        f\left(\bigvee_{i=1}^\infty a_i \right) &= \bigvee_{i=1}^\infty f(a_i)\\
        f\left(\bigwedge_{i=1}^\infty a_i\right) &= \bigwedge_{i=1}^\infty f(a_i).
    \end{align*}
    And furthermore, for any $a \in A$ we have $f(a') = f(a)'$, and $f(1) = 1$
    and $f(0) = 0$.
\end{definition}


The concrete case of a measurable function motivates the abstract case:
\begin{definition}
    Given two Boolean $\sigma$-algebras, $A$ and $B$, a measurable function 
    $f:A\rightarrow B$ is an algebra homomorphism, from $B$ to $A$.
\end{definition}


\section{Products of $\sigma$-algebras}
Given two concrete $\sigma$-algebras, $(X,\mathcal{A})$ and $(Y,\mathcal{B})$,
it is natural to want to find a compatibile $\sigma$-algebra on the
cartesian product $X \times Y$. This is described in the product sigma
algebra construction, which actually works for arbitrarily many factors.
The ``compatibility" is provided by the universal property. 
\begin{definition}
    Let $\{(X_i,\mathcal{A}_i)\}_{i\in I}$ be an indexed collection of concrete
    $\sigma$-algebras. Define,
    \begin{equation*}
        \prod_{i \in I} \mathcal{A}_i := \{ \prod_{i \in I} A_i\;:\;A_i \in \mathcal{A}_i,\text{ and }A_i = X_i\text{ for all but finitely many }i \in I\}
    \end{equation*}
    and then,
    \begin{equation*}
        \bigotimes_{i \in I} \mathcal{A}_i := \sigma\left(\prod_{i \in I} \mathcal{A}_i\right).
    \end{equation*}
    This is a $\sigma$-algebra on $\prod_{i\in I} X_i$.
    
    The pair $\left(\prod_{i \in I} X_i,\bigotimes_{i \in I} \mathcal{A}_i\right)$ is 
    called the product $\sigma$-algebra.
    
    Let the projection onto the $X_j$ factor be denoted $\pi_j:\prod_{j \in I} X_i \rightarrow X_i$
\end{definition}

The crucial theorem characterising the product is the \emph{universal property}, defined
and proved below.
\begin{proposition}
    Let $\{(X_i,\mathcal{A}_i)\}_{i \in I}$ be an indexed collection of $\sigma$-algebras.
    Let $(Y,\mathcal{B})$ be a $\sigma$-algebra, and suppose that for each $i \in I$, 
    there is a measurable function $f_i:Y\rightarrow X_i$. Then there exists a unique
    measurable function $f:Y\rightarrow \prod_{i \in I} X_i$, such that for each $i$
     the following diagram commutes.
    \begin{displaymath}
        \xymatrix{
           &
           (Y,\mathcal{B}) \ar@{.>}[d]^f \ar[ld]^{f_i}& \\
           (X_i,\mathcal{A}_i) & 
           \left(\prod_{i\in I} X_i,\bigotimes_{i \in I} \mathcal{A}_i\right)\ar[l]^{\pi_i}&     
        }
    \end{displaymath}
\end{proposition}
\begin{proof}
    Define the function $f = (f_i)_{i \in I}$. We need to show that $f^{-1}(C) \in \mathcal{B}$
    for each $\mathcal{C} \in \prod_{i \in I} \mathcal{A}_i$. So we can say that $C = \prod_{i \in I} A_i$
    for some selection $A_i \in \mathcal{A}_i$, where $A_i \neq X_i$ for only finitely
    many $i$. Then 
    \begin{equation*}
        f^{-1}(C) = \bigcap_{\{i \in I\;:\;A_i \neq X_i\}}f_{i}^{-1}(A_i) \in \mathcal{B}.
    \end{equation*}
\end{proof}

The product $\sigma$-algebra satisfies numerous nice properties. We document a
few now.
\begin{proposition}
    Suppose that $(X,\mathcal{A})$ and $(Y,\mathcal{B})$ are concrete $\sigma$-algebras.
    Let $E \subseteq X\times Y$. Then for each $x \in X$ and $y \in Y$,
    define
    \begin{align*}
        E^x &:= \{y \in Y\;:\;(x,y) \in E\}\\
        E^y &:= \{x \in X\;:\;(x,y) \in E\}.
    \end{align*}
    $E^x$ is called the $x$-slice of $E$, and $E^y$ is the $y$-slice of $E$. 
    If $E \in \mathcal{A} \otimes \mathcal{B}$, then $E^x \in \mathcal{B}$
    and $E^y \in \mathcal{A}$.
\end{proposition}
\begin{proof}
    It is sufficient to prove the following. Denote by $\mathcal{F}$
    the following set,
    \begin{equation*}
        \mathcal{F} = \{E \subseteq X\times Y\;:\;\forall (x,y) \in X\times Y, E^x \in \mathcal{B},E^y \in \mathcal{A}\}.
    \end{equation*}
    We wish to prove that $\mathcal{F} = \mathcal{A}\otimes \mathcal{B}$.
       
    We have $\mathcal{A} \times \mathcal{B} \subseteq \mathcal{F}$.
   
    If $A,B \in \mathcal{F}$, with $B \subseteq A$, then $A \setminus B \in \mathcal{F}$
    since $(A \setminus B)^x = A^x \setminus B^x$.
    
    If $\{A_n\}_{n=1}^\infty \subseteq \mathcal{F}$ is an increasing sequence,
    then $\left(\bigcup_{n=1}^\infty A_n\right)^x = \bigcup_{n=1}^\infty A_n^x \in \mathcal{F}$.
    
    Hence $\mathcal{F}$ is a $d$-class, so $d(\mathcal{A}\times\mathcal{B}) \subseteq \mathcal{F}$.
    
    However, $\mathcal{A}\times\mathcal{B}$ is a $\pi$-class, so
    by the monotone class theorem, we therefore have that $\mathcal{A}\otimes\mathcal{B} = \sigma(\mathcal{A}\times\mathcal{B}) \subseteq \mathcal{F}$.
    
\end{proof}


\begin{proposition}
    Suppose that $(X,\mathcal{A})$, $(Y,\mathcal{B})$ and $(E,\mathcal{E})$
    are $\sigma$-algebras.
    
    The a function $f:X\times Y\rightarrow E$ is $\mathcal{A}\otimes \mathcal{B}$-measurable
    if and only if, for every $x \in X$ and $y \in Y$, we have that
    the function $f^x:y\mapsto f(x,y)$ is $\mathcal{B}$-measurable
    and the function $f^y:x\mapsto f(x,y)$ is $\mathcal{A}$-measurable.
\end{proposition}
\begin{proof}
    First suppose that $f:X\times Y\rightarrow E$
    is $\mathcal{A}\otimes \mathcal{B}$-measurable. Then fix $x \in X$. Let $F \in \mathcal{E}$.
    
    Hence,
    \begin{equation*}
        (f^x)^{-1}(F) = \{y\;:\;f(x,y) \in F\}.
    \end{equation*}
    But the right hand side is simply,
    \begin{equation*}
        \{(x,y)\;:\;f(x,y) \in F\}^x = f^{-1}(F)^x.
    \end{equation*}
    However by assumption $f^{-1}(F) \in \mathcal{A}\otimes\mathcal{B}$. 
    Hence, $f^{-1}(F) \in \mathcal{B}$. Hence $f^x$ is $\mathcal{B}$-measurable.
    
    By symmetry, for each $y \in Y$ we have $f^y$ is $\mathcal{A}$-measurable
    
    
    
\end{proof}
\section{Real measurable functions}
We will now focus on measurable real valued functions on a concrete $sigma$-algebra.
This will motivate our definitions for measurable real valued functions
on an abstract $\sigma$-algebra. We shall use the machinery of product measures
here. 

The real numbers, $\Rl$, are given a $\sigma$-algebra, generated
by all the half-open intervals of the form $(a,b]$, for $a < b \in \Rl$. It is
easy to see that this is equivalently generated by rays, $(a,\infty)$, $[a,\infty)$
or open intervals $(a,b)$ or closed intervals $[a,b]$, or open sets.

Given a concrete $\sigma$-algebra $(X,\mathcal{A})$, let $\mathcal{M}(X,\mathcal{A})$
denote the set of measurable functions from $X$ to $(\Rl,\mathcal{B}(\Rl))$.

\begin{lemma}
    Let $X = \prod_{i=1}^\infty[0,\infty]$, where each factor has the Borel
    $\sigma$-algebra, and $X$ has the product algebra. Then the following functions
    are measurable:
    \begin{enumerate}
        \item{} $(x_n)_{n=1}^\infty \mapsto \sup_n x_n$
        \item{} $(x_n)_{n=1}^\infty \mapsto \inf_n x_n$
        \item{} $(x_n)_{n=1}^\infty \mapsto \limsup_n x_n$
        \item{} $(x_n)_{n=1}^\infty \mapsto \liminf_n x_n$.
    \end{enumerate}
\end{lemma}
\begin{proof}
    We only need to prove $1$. Let
    \begin{equation*}
        f((x_n)_{n=1}^\infty) = \sup_n x_n.
    \end{equation*}
    We compute $f^{-1}((a,\infty])$ for some $a > 0$. This is simply,
    \begin{equation*}
        f^{-1}((a,\infty]) = \bigcup_{n=1}^\infty \prod_{i=1}^{n-1} [0,\infty]\times (a,\infty])\times\prod_{i=n+1}^\infty [0,\infty)
    \end{equation*}
    Which is in the product $\sigma$-algebra.
\end{proof}

\begin{lemma}
    Let $X = \Rl^2$, given the $\sigma$-algebra $\mathbb{B}(\Rl)\otimes \mathbb{B}(\Rl)$.
    Then the function $(x,y)\mapsto x+y$ is measurable.
\end{lemma}
\begin{proof}
    This is true
    because addition is continuous. 
\end{proof} 


\begin{proposition}
    $\mathcal{M}(X,\mathcal{A})$ is a real vector space, when provided
    with pointwise function addition and scalar multiplication.
\end{proposition}
\begin{proof}
    Let $f \in \mathcal{M}(X,\mathcal{A})$. Given $\alpha \in \Rl$, if $\alpha = 0$,
    we clearly have $\alpha f \in \mathcal{M}(X,\mathcal{A})$, so assume $\alpha \neq 0$.
    
    First assume that $\alpha > 0$. We can easily compute, $(\alpha f)^{-1}((a,b]) = f^{-1}(a/\alpha,b/\alpha]) \in \mathcal{A}$.
    
    Now if $\alpha < 0$, we have $(\alpha f)^{-1}((a,b]) = f^{-1}([b/\alpha,a/\alpha)) \in \mathcal{A}$.
    
    Now if $f,g \in \mathcal{M}(X,\mathcal{A})$, we can compute $f+g$ as a composition,
    $x \mapsto (f(x),g(x))\mapsto f(x) + g(x)$. This is a composition of measurable maps.
\end{proof}

\begin{proposition}
    Given a sequence $\{f_n\}_{n=1}^\infty \subseteq \mathcal{M}(X,\mathcal{A})$.
    Then $\inf_{n} f_n ,\sup_n f_n \in \mathcal{M}(X,\mathcal{A})$.
\end{proposition}
\begin{proof}
    Since each $f_n$ is measurable, the product map $x\mapsto (f_n(x))_{n\in\Ntrl}$
    to $\Rl^\Ntrl$ is measurable. 
\end{proof}

Now, we use these proofs to motivate our construction of $\mathcal{M}(A)$,
where $A$ is a Boolean $\sigma$-algebra.

\section{Measure Algebras}
Now that we have all the order theory out of the way, we can finally
get to the theory of measures. The elementary concept
that we introduce here is a \emph{measure algebra}.
\begin{definition}
    A measure algebra is a pair $(A,\mu)$, where $A$ is a Boolean $\sigma$-algebra and
     $\mu:A\rightarrow[0,\infty]$ is such that:
    \begin{enumerate}
        \item{} $\mu(a) < \infty$ for some $a \in A$.
        \item{} If $\{a_i\}_{i=1}^\infty \subseteq A$ is a sequence that is
        \emph{pairwise disjoint} (meaning $a_i \wedge a_j = 0$ for $i\neq j$)
        we have
        \begin{equation*}
            \mu\left(\bigvee_{i=1}^\infty a_i\right) = \sum_{i=1}^\infty \mu(a_i).
        \end{equation*}
    \end{enumerate}
\end{definition}
\begin{remark}
    Given a $sigma$-algebra $\mathcal{A}$ of subsets of a set $X$, and a measure
    algebra $(\mathcal{A},\mu)$, the triple $(X,\mathcal{A},\mu)$ is called a measure
    space. Measure spaces are special cases of measure algebras, so we prove
    as many things as possible for measure algebras before later on specialising
    to measure spaces.
\end{remark}
\begin{proposition}
    Let $(A,\mu)$ be a measure algebra. Then $\mu(0) = 0$. 
\end{proposition}
\begin{proof}
    Let $a \in A$ with $\mu(a) < \infty$. By countable disjoint additivity, we have
    \begin{equation*}
        \mu(a\vee 0 \vee 0 \vee 0 \vee \cdots) = \mu(a) + \mu(0) + \mu(0) + \cdots.
    \end{equation*}
    Hence,
    \begin{equation*}
        \mu(0) + \mu(0) + \mu(0) + \cdots = 0.
    \end{equation*}
    Thus $\mu(0) = 0$.
\end{proof}
\begin{proposition}
    Measure are finitely disjointly additive. That is, if $(A,\mu)$ is a measure
    algebra, and $a,b \in A$ with $a \wedge b = 0$, then $\mu(a \vee b) = \mu(a) + \mu(b)$.
\end{proposition}
\begin{proof}
    Write $a \vee b = a \vee b \vee 0 \vee 0 \vee 0 \vee \cdots$, then use
    countable disjoint additivity and $\mu(0) = 0$.
\end{proof}
\begin{proposition}
    Let $(A,\mu)$ be a measure algebra, and $a,b \in A$. Then,
    \begin{equation*}
        \mu(a \vee b) + \mu(a \wedge b) = \mu(a) + \mu(b).
    \end{equation*}
    If $\mu(a \wedge b) < \infty$, we have the familiar inclusion-exclusion principle,
    $\mu(a\vee b) = \mu(a) + \mu(b) - \mu(a \wedge b)$.
\end{proposition}
\begin{proof}
    We can write, $a \vee b = a \vee ( b \wedge a')$, and $a \vee b = (a \wedge b) \vee (a \wedge b') \vee (b \wedge a')$. 
    Then by disjoint additivity, we have
    \begin{equation*}
        \mu(a \vee b) = \mu(a\wedge b) + \mu(a \wedge b') + \mu(a' \wedge b).
    \end{equation*}
    Hence,
    \begin{equation*}
        \mu(a \vee b)+\mu(a\wedge b) = (\mu(a\wedge b) + \mu(a \wedge b')) + (\mu(a \wedge b) + \mu(a'\wedge b))
    \end{equation*}
    But by disjoint additvity, the right hand side is simply $\mu(a) + \mu(b)$.
    
\end{proof} 
\begin{proposition}
    Suppose that $(A,\mu)$ is a measure algebra, and $\{a_i\}_{i=1}^\infty$
    is an increasing sequence of elements of $A$, that is, $a_i \leq a_j$
    for $i \leq j$. Then
    \begin{equation*}
        \mu\left(\bigvee_{i=1}^\infty a_i\right) = \lim_{i\rightarrow\infty} \mu(a_i).
    \end{equation*}
\end{proposition}
\begin{proof}
    We ``disjointify" the sequence $\{a_i\}_{i=1}^\infty$. Let $b_i = a_i \wedge a_{i-1}'$
    for $i>1$. Then $\bigvee_{i=1}^\infty b_i = \bigvee_{i=1}^\infty a_i$, but 
    the sequence $\{b_i\}_{i=1}^\infty$ is pairwise disjoint. Hence,
    \begin{align*}
        \mu\left(\bigvee_{i=1}^\infty a_i\right) &= \mu\left(\bigvee_{i=1}^\infty b_i\right)\\
        &= \sum_{i=1}^\infty \mu(b_i)\\
        &= \lim_{n\rightarrow\infty} \sum_{i=1}^n \mu(b_i)\\
        &= \lim_{n\rightarrow \infty} \mu\left(\bigvee_{i=1}^n b_i\right)\\
        &= \lim_{n\rightarrow\infty} \mu(a_n).
    \end{align*}
\end{proof}
\begin{proposition}
    Let $(A,\mu)$ be a measure algebra. Let $\{a_i\}_{i=1}^\infty$ be a decreasing
    sequence of elements of $a$, that is $a_i \leq a_j$ for $i \geq j$, furthermore
    assume that $\mu(a_1) < \infty$. Then
    \begin{equation*}
        \mu\left(\bigwedge_{i=1}^\infty a_i\right) = \lim_{n\rightarrow\infty} \mu(a_n).
    \end{equation*}
\end{proposition}
\begin{proof}
    We have,
    \begin{align*}
        \mu\left(a_1\wedge \left(\bigwedge_{i=1}^\infty a_i\right)'\right) &= \mu\left(\bigvee_{i=1}^\infty a_1\wedge a_i' \right)\\
        &= \lim_{n\rightarrow\infty}  \mu(a_1\wedge a_n')\\
        &= \lim_{n\rightarrow\infty} \mu(a_1)-\mu(a_n).
    \end{align*}
    Hence, subtract $\mu(a_1)$ from both sides and the result follows.
\end{proof}

There is an important property of measure spaces, called $\sigma$-finiteness.
\begin{definition}
    A measure space $(X,\mathcal{A},\mu)$ is $\sigma$-finite if there is
    a sequence $\{A_n\}_{n=1}^\infty \subseteq \mathcal{A}$ such
    that $\mu(A_n) < \infty$ for each $n$, but $\bigcup_{n=1}^\infty A_n = X$.
\end{definition}

\begin{proposition}
    Suppose that $(X,\mathcal{A})$ is a $\sigma$-algebra, where $\mathcal{A} = \sigma(\mathcal{C})$
    for some $\pi$-class $\mathcal{C}$. If two $\sigma$-finite measures $\mu$ and $\nu$ on $\mathcal{A}$
    agree on $\mathcal{C}$, then they are equal.
\end{proposition}
\begin{proof}
    First consider the case where $\mu$ and $\nu$ are finite. Then, let $\mathcal{F}$
    be the collection of all $F \in \mathcal{A}$ such that $\mu(A) = \nu(A)$. We see
    that if $A,B \in \mathcal{F}$, with $B \subseteq A$, then 
    \begin{equation*}
        \mu(A\setminus B) = \mu(A)-\mu(B) = \nu(A)-\nu(B) = \nu(A\setminus B).
    \end{equation*}
    So $A\setminus B \in \mathcal{F}$.
    
    Now if $\{A_n\}_{n=1}^\infty$ is an increasing sequence in $\mathcal{F}$, we
    have 
    \begin{equation*}
        \mu\left(\bigcup_{n=1}^\infty A_n\right) = \lim_{n\rightarrow\infty} \mu\left(A_n\right) = \lim_{n\rightarrow\infty} \nu(A_n) = \nu\left(\bigcup_{n=1}^\infty\right)
    \end{equation*} 
    
    Hence $\mathcal{F}$ is a $d$-class, so $\mathcal{A} = \sigma(\mathcal{C}) = d(\mathcal{C}) \subseteq \mathcal{F}$.
    
    Now assume that $\mu$ and $\nu$ are $\sigma$-finite. We can find an
    increasing sequence $\{A_n\}_{n=1}^\infty$ such that $\mu(A_n),\nu(A_n) <\infty$
    and $X = \bigcup_{n=1}^\infty A_n$. Then, if $A \in \mathcal{F}$ since the
    measures $\mu_n(A) = \mu(A_n\cap A$ and $\nu_n(A) = \nu(A\cap A_n)$, we have
    \begin{equation*}
        \mu(A) = \lim_{n\rightarrow\infty} \mu(A\cap A_n) = \lim_{n\rightarrow\infty} \nu(A\cap A_n) = \nu(A).
    \end{equation*}
\end{proof}

\section{Integration of non-negative functions}
Let $X$ be any set, and let $f:X\rightarrow [0,\infty)$. We can find a sequence
of functions of finite range that converges to $f$ pointwisely and monotonically
from below.

Let 
\begin{equation*}
    s_n = \sum_{k=1}^{2^{2n}} \frac{k}{2^n}\chi_{f^{-1}([k/2^n,(k+1)/2^n))}
\end{equation*}

Now let $(X,\mathcal{A},\mu)$ be a measure space. A function $s:X\rightarrow[0,\infty]$
is called simple if it measurable and has finite image. If
\begin{equation*}
    s = \sum_{k=1}^n a_k \chi_{A_k}
\end{equation*}
for some disjoint selection $A_k \in \mathcal{A}$, we define the integral of $s$
to be
\begin{equation*}
    \int_X s\;d\mu := \sum_{k=1}^n a_k\mu(A_k)
\end{equation*}
where we assert only for the purposes of this definition that $0\cdot\infty = 0$.

Now if $f:X\rightarrow [0,\infty]$ is any measurable function, we define
\begin{equation*}
    \int_X f\;d\mu = \sup\{\int_X s\;d\mu\;:\;s\text{ is a simple function },s \leq f\}
\end{equation*}

Our first proposition is obvious,
\begin{proposition}
    Let $f,g:X\rightarrow[0,\infty]$ be measurable, and $f \leq g$, then
    \begin{equation*}
        \int_X f\;d\mu \leq \int_X g\;d\mu.
    \end{equation*}
\end{proposition}
\begin{proof}
    This follows from the fact 
    \begin{equation*}
        \{s\;:\;\text{ is a simple function },s \leq f\}
    \end{equation*}
    is a subset of
    \begin{equation*}
        \{s\;:\;\text{ is a simple function },s \leq g\}
    \end{equation*}
\end{proof}

\begin{proposition}
    Suppose that $\{f_n\}_{n=1}^\infty$ is a monotonically increasing
    sequence of measurable functions from $X$ to $[0,\infty)$. Then
    \begin{equation*}
        \lim_{n\rightarrow\infty} \int_Xf_n\;d\mu = \int_X \lim_{n\rightarrow\infty} f_n\;d\mu.
    \end{equation*}
\end{proposition} 
\begin{proof}
    Let $f = \lim_{n\rightarrow \infty} f_n$. For any $n$, we have $f_n\leq f$. Hence,
    \begin{equation*}
        \int_{X} f_n\;d\mu \leq \int_X f\;d\mu.
    \end{equation*}
    Thus,
    \begin{equation*}
        \lim_{n\rightarrow\infty} \int_X f_n\;d\mu \leq\int_X f\;d\mu.
    \end{equation*}
    
    So we must prove the reverse inequality.
    
    Let $s$ be a simple function with $s \leq f$.
    Let $s = \sum_{k=1}^N s_k \chi_{B_k}$, where the $B_k$
    are pairwise disjoint.
     We would like to show that
    $\lim_{n\rightarrow\infty} f_nd\mu \geq \int_X s\;d\mu$. To this end,
    let $c \in (0,1)$. Define
    \begin{equation*}
        A_n = \{x\;:\;f_n \geq cs\}.
    \end{equation*}
    We have,
    \begin{equation*}
        f_n \geq f_n \chi_{A_n} \geq cs\chi_{A_n}.
    \end{equation*} 
    Hence,
    \begin{equation*}
        \int_X f_n\;d\mu \geq \sum_{k=1}^N cs_k \mu(B_k \cap A_n).
    \end{equation*}
    Now we take the limit. Note that since $f_n$ converges pointwisely,
    $\bigcup_n A_n = X$. Thus, $\lim_{n\rightarrow\infty} \mu(B_k \cap A_n) = \mu(B_k)$.
    
    Thus, 
    \begin{equation*}
        \lim_{n\rightarrow\infty} \int_X f_n\;d\mu \geq c\int_X s\;d\mu.
    \end{equation*}
    But $c$ is arbitrary, so we have our result.
\end{proof} 
  
  
The next result is called Fatou's lemma. 
\begin{lemma}
    Let $\{f_n\}_{n=1}^\infty$ be a sequence of measurable functions from $X$ to $[0,\infty)$. We have
    \begin{equation*}
        \int_X \liminf_{n} f_n\;d\mu \leq \liminf_{n} \int_X f_n\;d\mu.
    \end{equation*}
\end{lemma}
\begin{proof}
    For any $k \geq 0$, and $j\geq n$, we have
    \begin{equation*}
        \inf_{n\geq k} f_n \leq f_j.
    \end{equation*}
    Hence, we integrate this,
    \begin{equation*}
        \int_X \inf_{n\geq k} f_n \leq \int_{X} f_j\;d\mu.
    \end{equation*}
    But since $j \geq k$ is arbitrary, this implies
    \begin{equation*}
        \int_X \inf_{n\geq k} f_n\;d\mu \leq \inf_{n\geq k} \int_X f_n \;d\mu.
    \end{equation*}
    Hence,
    \begin{equation*}
        \lim_{k\rightarrow\infty} \int_X \inf_{n\geq k} f_n\;d\mu \leq \lim_{k\rightarrow\infty} \inf_{n\geq k} \int_X f_n\;d\mu.
    \end{equation*}
    So by the monotone convergence theorem,
    \begin{equation*}
        \int_X \liminf_n f_n\;d\mu \leq \liminf_n \int_X f_n\;d\mu.
    \end{equation*}
\end{proof}
  
The next theorem generalises the monotone convergence theorem, and is called
the convergence from beneath theorem:
\begin{proposition}
    Suppose that $\{f_n\}_{n=1}^\infty$ is sequence of measurable
    functions from $X$ to $[0,\infty]$. Suppose that $\lim_{n\rightarrow\infty} = f$,
    and for each $n$, we have $f_n \leq f$. Then
    \begin{equation*}
        \lim_{n\rightarrow\infty} \int_X f_n\;d\mu = \int_X f\;d\mu.
    \end{equation*}
\end{proposition}
\begin{proof}
    For each $n$, we have $f_n \leq f$, hence
    \begin{equation*}
        \limsup_n \int_X f_n \;d\mu \leq \int_X f\;d\mu.
    \end{equation*}
    But also $\liminf_n f_n = f$, so by Fatou's lemma,
    \begin{equation*}
        \int_X f\;d\mu = \int_X \liminf_n f_n\;d\mu \leq \liminf_n \int_X f_n\;d\mu.
    \end{equation*}
    So we are done.
\end{proof}


The monotone convergence theorem justifies our definition of the integral
in the case of abstract measure algebras. Suppose that $(A,\mu)$ is a measure algebra,
and $f:A\rightarrow \mathcal{B}([0,\infty])$ is a measurable function (that is,
$f$ is a $\sigma$-algebra homomorphism from $\mathcal{B}([0,\infty])$ to $A$. Then define
\begin{equation*}
    \int_X f\;d\mu := \lim_{n\rightarrow\infty} \sum_{k=1}^{2^{2n}} \frac{k}{2^n}\mu(f(k/2^n,(k+1)/2^n])
\end{equation*}

Now using this definition, we prove some further properties.
\begin{proposition}
    Let $(A,\mu)$ be a $\sigma$-algebra. Let $f,g:A\rightarrow\mathcal{B}([0,\infty])$
\end{proposition}

\section{Product measures and Tonelli's theorem}
Let $(X,\mathcal{A},\mu)$ and $(Y,\mathcal{B},\nu)$ be $\sigma$-finite measure
spaces. We would like to define a compatible measure $\mu\times\nu$ on $(X\times Y,\mathcal{A}\otimes \mathcal{B})$. 

We can specify this uniquely by saying that for each $A\times B \in \mathcal{A}\times \mathcal{B}$,
we have $(\mu\times \nu)(A\times B) = \mu(A)\nu(B)$. This determines
the measure uniquely, since it is obvious that any measure satisfying this equality
is $\sigma$-finite, and $\mathcal{A}\times\mathcal{B}$ is a $\pi$-class.

So we only need to prove that $\mu\times\nu$ exists. First we need to
prove the following,
\begin{lemma}
    Let $E \in \mathcal{A}\otimes\mathcal{B}$. Then the function $X\rightarrow [0,\infty]$
    given by $x\mapsto \nu(E^x)$ is $\mathcal{A}$ measurable.
\end{lemma}
\begin{proof}
    
    Let $\mathcal{F}$ be the set of all $E \subseteq X\times Y$ such that $x \mapsto \nu(E^x)$
    is $\mathcal{A}$-measurable. 
    
    We clearly have $\mathcal{A}\times\mathcal{B} \subseteq \mathcal{F}$.
    
    Suppose that $\{A_n\}_{n=1}^\infty$ is an increasing sequence in $\mathcal{F}$. Then $\nu\left(\bigcup_{n=1}^\infty A_n\right)^x = \nu\left(\bigcup_{n=1}^\infty A_n^x\right)$.
    But by countable additivity, this is $\sum_{n=1}^\infty \nu(A_n^x)$.
    This is a supremum of measurable functions, hence is measurable.
    
    Now assume that $\nu$ is a finite measure.
    
    If $A,B \in \mathcal{F}$, then we have $\nu(A\setminus B)^x = \nu(A^x\setminus B^x) = \nu(A^x) - \nu(B^x)$,
    hence $A \setminus B \in \mathcal{F}$.
    
    Now if $\nu$ is $\sigma$-finite, we have for each $B \in \mathcal{B}$, a sequence
    of sets of finite measure $A_n$ such that $\nu(B) = \lim_{n\rightarrow\infty} \nu(A_n\cap B)$. 
    
    Hence, if $A,B \in \mathcal{F}$, then
    \begin{align*}
        \nu(A\setminus B)^x &= \lim_{n\rightarrow\infty}\\
        &=  \nu((A\cap A_n)^x\setminus (B\cap A_n)^x)\\
         &= \lim_{n\rightarrow\infty} \nu(A\cap A_n)^x-\nu(B\cap A_n)^x\\
         &= \lim_{n\rightarrow\infty} \nu(A_n\cap(A\setminus B))^x\\
    \end{align*}
    Hence, $A\setminus B \in \mathcal{F}$.
    
\end{proof}

\begin{proposition}
    Let $(X,\mathcal{A},\mu)$ and $(Y,\mathcal{B},\nu)$ be $\sigma$-finite
    measure spaces. Then there exists a product measure on $(X\times Y,\mathcal{A}\otimes \mathcal{B})$.
\end{proposition}
\begin{proof}
    Define, for each $E \in \mathcal{A}\otimes \mathcal{B}$,
    \begin{equation*}
        (\mu\times\nu)(E) := \int_X \nu(E^x)\;d\mu(x).        
    \end{equation*}
    This is a measure by the monotone convergence theorem. That is, if $\{A_n\}_{n=1}^\infty$
    is a pairwise disjoint subset of $\mathcal{A}\otimes\mathcal{B}$, then
    \begin{equation*}
        (\mu\times\nu)\left(\bigcup_{n=1}^\infty A_n\right) = \sum_{n=1}^\infty \int_{X} \nu(A_n^x)\;d\mu(x).
    \end{equation*}
    and we have $(\mu\times\nu)(\emptyset) = 0$
\end{proof}

Now we prove the famous and important Tonelli's theorem:
\begin{proposition}
    Let $(X,\mathcal{A},\nu)$ and $(Y,\mathcal{B},\mu)$ be measure spaces,
    and let $f:X\times Y\rightarrow[0,\infty]$ be $\mathcal{A}\otimes\mathcal{B}$
    measurable. Then
    \begin{equation*}
        \int_{X\times Y} f\;d(\mu\times\nu) = \int_Y \int_X f(x,y)\;d\mu(x)d\nu(y) = \int_X \int_Y f(x,y)\; d\nu(y)d\mu(x)
    \end{equation*}
\end{proposition}
\begin{proof}
    This is true by definition when $f$ is an indicator function. Hence by linearity
    it is true when $f$ is simple. Then by the monotone convergence theorem,
    it is true for any non-negative measurable function $f$.
\end{proof}

\section{$L^p$ spaces}
So far we have dealt only with non-negative measurable functions. We define
spaces of $\Cplx$ valued functions by noting that for any $p > 0$, the function
$x\mapsto |x|^p$ is measurable on $\Cplx$ with the Borel $\sigma$-algebra.

Hence define,
\begin{definition}
    Given $p > 0$ and a measure space $(X,\mathcal{A},\mu)$, define the set
    \begin{equation*}
        \mathcal{L}^p(X,\mu) := \{f:X\rightarrow\Cplx\;:\;f\text{ is }\mathcal{B}(\Cplx)\text{ measurable and},\int_X |f|^p\;d\mu < \infty\}
    \end{equation*}
\end{definition}

Note that for any non-negative real numbers $a$ and $b$, we have the inequaluty $(a+b)^p \leq a^p + b^p$.
Hence, $\mathcal{L}^p(X,\mu)$ is a complex vector space. Define the quantity,
\begin{equation*}
    \|f\|_p = \left(\int_X |f|^p\;d\mu\right)^{1/p}
\end{equation*}
Hence we have $\|f+g\|_p^p \leq \|f\|_p^p + \|g\|_p^p$.

Given $f \in \mathcal{L}^1(X,\mu)$, we can write $f$
as $f = (\max\{\Re(f),0\}-\max\{0,-\Re{f}\}) + i(\max\{\Im(f),0\}-\max\{0,-\Im(f)\})$,
then define
\begin{equation*}
    \int_X f\;d\mu = \int_X \max\{\Re(f),0\}\;d\mu - \int_X \max\{0,-\Re(f)\}\;d\mu + i\left(\int_X \max\{\Im(f),0\}\;d\mu - \int_X \max\{0,-\Im(f)\}\;d\mu\right)
\end{equation*}

Now we can state the important dominated convergence theorem,
\begin{proposition}
    Let $\{f_n\}_{n=1}^\infty \subset \mathcal{L}^1(X,\mu)$ be such that
    there exists a function $g \in \mathcal{L}^1(X,\mu)$ with $|f_n| \leq g$
    for all $n$. Suppose that for every $x$, $f_n(x)\rightarrow f(x)$. Then
    $\|f_n-f\|_1\rightarrow 0$, and hence
    \begin{equation*}
        \lim_{n\rightarrow\infty} \int_X f_n \;d\mu = \int_X f\;d\mu.
    \end{equation*}
\end{proposition}
\begin{proof}
    We have $|f-f_n| \leq 2g$, so by Fatou's lemma,
    \begin{equation*}
        \int_X \liminf_{n} 2g-|f-f_n| \;d\mu \leq \liminf_n\int_X 2g-|f-f_n|\;d\mu.
    \end{equation*}
    Hence,
    \begin{equation*}
        \limsup_n \int_X |f-f_n|\;d\mu \leq 0.
    \end{equation*}
\end{proof}

Crucial to defining $L^p$ spaces is the following inequality,
called the H\"older inequality.

First we prove a lemma,
\begin{lemma}
    Let $a,b \geq 0$, and $p > 1$ with $1/p+1/q = 1$. Then
    \begin{equation*}
        ab \leq \frac{a^p}{p}+\frac{b^q}{q}
    \end{equation*}
\end{lemma}
\begin{proof}
    Consider the rectanble $[0,a]\times [0,b]$. The area graph of the function $x\mapsto x^{p-1}$
    on the interval $[0,a]$ covers some fraction of $[0,a]\times [0,b]$. The remaining
    part of the rectangle is covered by the graph if the inverse function, $x\mapsto x^{\frac{1}{p-1}}$.
    Hence,
    \begin{equation*}
        ab\leq \int_0^a x^{p-1}\;dx + \int_0^{b} x^{\frac{1}{p-1}}\;dx = \frac{a^p}{p}+\frac{b^q}{q}.
    \end{equation*}
\end{proof}
\begin{proposition}
    Suppose $p \geq 1$, and $1/p + 1/q = 1$. Then,
    \begin{equation*}
        \|fg\|_1 \leq \|f\|_p\|g\|_q.
    \end{equation*}
\end{proposition}
\begin{proof}
    If $\|f\|_p = 0$ or $\|g\|_p = 0$, the result is easy. So assume
    that $\|f\|_p = \|g\|_p = 1$. Then we have, for each $x \in X$,
    \begin{equation*}
        |f(x)g(x)| \leq \frac{|f(x)|^p}{p}+\frac{|g(x)|^q}{q}
    \end{equation*}
    Integrate over $p$ to obtain the result.
\end{proof}

The next result is called Minkowski's inequality,
\begin{proposition}
    Let $p\geq 1$. Then $\|f+g\|_p \leq \|f\|_p + \|g\|_p$.
\end{proposition}
\begin{proof}
    Assume $\|f+g\|_p \neq 0$, because then it is obvious. 
    
    We compute,
    \begin{equation*}
        \|f+g\|_p^p \leq \int_{X} |f||f+g|^{p-1}\;d\mu + \int_X |g||f+g|^{p-1}\;d\mu
    \end{equation*}
    
    Now we use H\"older's inequality,
    \begin{equation*}
        \|f+g\|_p^p \leq (\|f\|_p+\|g\|_p)\frac{\|f+g\|_p^p}{\|f+g\|_p}.
    \end{equation*}
    And the result follows.
\end{proof}

Now we introduce the $L^p$ spaces. Define an equivalence
relation on $\mathcal{L}^p(X,\mu)$ such that two
functions are equivalent if they differ on a set of measure $0$.
This space is called $L^p(X,\mu)$. The algebraic operations and norms
on $\mathcal{L}^p(X,\mu)$ are well defined for $L^p(X,\mu)$

\begin{proposition}
    Let $p > 0$. Then when equipped with the metric,
    $d_p(f,g) = \|f-g\|_p^p$, $L^p(X,\mu)$ is a complete metric space.
\end{proposition}
\begin{proof}
    If we suppose that $L^p(X,\mu)$ is complete, then given every series
    $\sum_n u_n$ such that $\sum_n \|u_n\|_p^p$ converges, then $\sum_{n} u_n$
    converges. This follows from the bound,
    \begin{equation*}
        \|u_n + u_{n+1} + u_{n+2} + \cdots + u_{n+m}\|^p_p \leq \|u_n\|_p^p + \|u_{n+1}\|_p^p + \cdots + \|u_{n+m}\|_p^p.
    \end{equation*}
    Conversely, if $L^p(X,\mu)$ has the property that every series of the form
    $\sum_{n} u_n$ where $\sum_{n} \|u_n\|_p^p$ converges, then $\sum_{n} u_n$
    converges in the $L^p$ space, then we prove that $L^p(X,\mu)$ is complete.
    
    Let $\{u_n\}_{n\in \Ntrl}$ be a Cauchy sequence in the $d_p$ metric. Choose
    a subsequence $\{u_{\varphi(n)}\}_{n\in\Ntrl}$ where $\varphi:\Ntrl\rightarrow\Ntrl$
    is increasingly rapidly enough such that $d_p(u_{\varphi(n)},u_{\varphi(n+1)}) < 2^{-n}$. 
    Then the series $\sum_{n} u_{\varphi(n+1)}-u_{\varphi(n)}$ converges. Hence $\{u_n\}_n$
    converges, so $L^p(X,\mu)$ is complete.
    
    
    Hence to prove that $L^p(X,\mu)$ is complete, it is sufficient to prove that every
    series of the form $\sum_n u_n$ where $\sum_n \|u_n\|_p^p$ converges
    is convergent. By the monotone convergence theorem, we have
    
    \begin{equation*}
        \lim_{n\rightarrow\infty} \int_X \left(\sum_{k=1}^n |u_k|\right)^p\;d\mu = \int_X \left(\sum_{k=1}^\infty|u_k|\right)^p\;d\mu \leq \sum_{k=1}^\infty \|u_k\|_p^p.
    \end{equation*}
    
    Hence, we have an almost everywhere pointwise converging sum $\sum_{n} |u_n|$. Hence
    almost everywhere, the function $f = \sum_{n} u_n$ is defined. The functions,
    $f_N = \left(\sum_{n=1}^N u_n\right)^p$ are pointwise dominated by $G = \left(\sum_{n} |u_n|\right)^p$,
    which is $L^1$. Hence $f$ is in $L^1$, and the functions $g_n = \sum_{k=n}^\infty u_n$
    are dominated $G$ also, so $f_n\rightarrow f$ in the $L^p$ sense.
\end{proof}

\section{Signed Measures and Radon-Nykodym}
A simple generalisation of the measure concept is provided
by the idea of a signed measure.
\begin{definition}
    Let $A$ be a Boolean $\sigma$-algebra. A function $\nu:A\rightarrow [-\infty,\infty]$
    is called a signed measure if,
    \begin{enumerate}
        \item{} $\nu(0) = 0$.
        \item{} The values $-\infty$ and $+\infty$ are both not simultaneously obtained.
        \item{} For any sequence $\{a_k\}_{k=1}^\infty \subseteq A$ with $a_k \wedge a_j = 0$
        for all $k\neq j$, we have
        \begin{equation*}
            \nu\left(\bigvee_{i=1}^\infty a_i\right) = \sum_{i=1}^\infty \nu(a_i).
        \end{equation*}
    \end{enumerate}
\end{definition}

\begin{definition}
    An element $p$ of a signed measure algebra $(A,\nu)$ is called \emph{hereditarily positive}
    if for any $e \leq p$, we have $\nu(e) \geq 0$. Similarly an element $n$
    is called \emph{hereditarily negative} if for any $f \leq n$ we have $\nu(f) \leq 0$.
\end{definition}
Note that the elements which are both hereditarily negative and hereditarily
positive are exactly null sets.

\begin{lemma}
\label{positiveImpliesContainsPositive}
Let $(A,\nu)$ be a signed measure algebra, suppose that $a \in A$ has $\nu(A) > 0$.
Then there exists a hereditarily positive element $p \leq a$ with $\nu(p) > 0$.
\end{lemma}
\begin{proof}
    If $a$ is hereditarily positive, then we are done.
    
    Otherwise, there is some $a_1 < a$ with $\nu(a_1) < 0$.
    Choose $n_1$ as the smallest positive integer such that $\nu(a_1) < -1/n_1$.
    
    Now consider $a \wedge a_1'$. We must have $\nu(a\wedge a_1') > 0$. 
    If $a \wedge a_1'$ is hereditarily positive, we are done. 
    
    Otherwise, there is some $a_2 < a \wedge a_1'$ with $\nu(a_2) < 0$.
    
    Choose $n_2$ as the smallest positive integer such that $\nu(a_2) < -1/n_2$.
    
    Continue inductively, building elements $a_k \leq a\wedge \bigwedge_{i=1}^{k-1} a_i'$,
    with $n_k$ the smallest positive integer such that $\nu(a_k) \leq -1/n_k$.
    
    If no such $n_k$ exists, then we are done, and we can choose $p = a \wedge \bigwedge_{i=1}^{k-1} a_i'$.
    
    Otherwise, the process continues indefinitely. Then define,
    \begin{equation*}
        p := a \wedge \bigwedge_{i=1}^\infty a_i' = a\wedge\left(\bigvee_{i=1}^\infty a_i\right)'.
    \end{equation*}
    
    Note that we have,
    \begin{equation*}
        \nu(p) = \nu(a) - \sum_{i=1}^\infty \nu(a_i).
    \end{equation*}
    Hence we cannot have $\sum_i \nu(a_i) = -\infty$. Thus
    the sum $\sum_i 1/n_i$ converges. 
    
    Therefore the sequence $n_i \rightarrow \infty$ as $i\rightarrow\infty$.
    
    Now let $e \leq p$. If $\nu(e) < 0$, there is some $n_i$
    with $\nu(e) < -1/n_i$. Choose $k > i$ such that $n_k > n_i$. But
    then $e_k$ was chosen so that $n_k$ is the least integer such that $\nu(e_k) < -1/n_k$.
    
    However $\nu(e_k \vee e) < -1/n_i$, so $n_k$ cannot be minimal.
    
    This is a contradiction, so we must have $\nu(e) \geq 0$. Hence $p$
    is hereditarily positive.
\end{proof}

Now we prove the Hahn decomposition for measure algebras:
\begin{proposition}
    Suppose that $(A,\nu)$ is a signed measure algebra. Then there
    exists $p,n \in A$ with $p \vee n = 1$, $\nu(p \wedge n) = 0$
    and $p$ hereditiarily positive and $n$ hereditarily negative.
\end{proposition}
\begin{proof}
    Assume without loss of generality that $\nu$ does not take the value $+\infty$.
    
    Define
    \begin{equation*}
        \beta = \sup\{\nu(a)\;:\;a\text{ is hereditarily positive.}\}.
    \end{equation*}
    We see that $\beta$ exists because $0$ is hereditarily positive. Suppose
    that $\{a_k\}_{k=1}^\infty$ is a sequence with $\nu(a_k)\rightarrow\beta$.
    
    Since the join of any two hereditarily positive elements is
    hereditarily positive, we may take $a_k$ to be an increasing sequence. 
    Let 
    \begin{equation*}
        p := \bigvee_{k=1}^\infty a_k.
    \end{equation*}
    See that if $e \in A$, we have 
    \begin{equation*}
        \nu(p\wedge e) = \lim_{k\rightarrow\infty} \nu(a_k\wedge e) \geq 0.
    \end{equation*}
    So $p$ is hereditarily positive, and $\nu(p) \leq \beta$. But since
    \begin{equation*}
        \nu(p) = \lim_{k\rightarrow\infty} \nu(a_k)
    \end{equation*}
    we must have $\nu(p) = \beta$.
    
    Now let $n = p'$. Let $f \leq n$. Then if $\nu(f) > 0$ by lemma \ref{positiveImpliesContainsPositive}, we must have
    \begin{equation*}
        \nu(p\vee f) > \nu(p) > \beta
    \end{equation*}
    and $p \vee f$ is hereditarily positive, so this is impossible. 
    
    Hence $\nu(f) \leq 0$, and so $n$ is hereditarily negative.
    
    Since $p \wedge n \leq n$ and $p \wedge n \leq p$, $p\wedge n$
    is both hereditarily positive and negative, hence null.
\end{proof}

Two positive measures $\mu_1$
and $\mu_2$ are said to be mutually singular if $\mu_1(a) > 0$
implies that $\mu_2(a) = 0$, and vice versa. We write $\mu_1 \bot \mu_2$.

This allows us to state and prove the Jordan decomposition:
\begin{proposition}
    Let $(A,\nu)$ be a signed measure algebra. There exist unique positive
    measures $\nu_+$ and $\nu_-$ with $\nu_+\bot \nu_-$
    and $\nu = \nu_+-\nu_-$.
\end{proposition}
\begin{proof}
    Let $p,n$ be the Hahn decomposition of $A$. Define $\nu_+(a) = \nu(a\wedge p)$
    and $\nu_-(a) = -\nu(a\wedge n)$. 
    
\end{proof}

\begin{lemma}
    Suppose that $A$ is a Boolean $\sigma$-algebra, and $\lambda$ and $\mu$
    are two finite measures where $\lambda$ is positive, and which are not mutually singular. Then there
    exists an $\varepsilon > 0$ and an element $e \in A$ such that $e$
    is hereditarily positive for $\lambda-\varepsilon\mu$, and $\mu(e) > 0$.
\end{lemma}
\begin{proof}
    Define the measure $\nu_k = \lambda-(1/k)\mu$. Let $1 = p_k\vee n_k$
    be the associated Hahn decomposition. Then let $p = \bigvee_k p_k$
    and $n = p' = \bigwedge_k n_k$.
    
    Then $n$ is negative for every $\nu_k$, thus $\lambda(n) \leq 0$. Hence $\lambda(n) = 0$.
    
    If $\mu(p) = 0$, then $\mu$ and $\lambda$ are mututally singular.
    Hence $\mu(p_k) > 0$ for some $k$. Thus we have a solution
    for $\varepsilon = 1/k$ and $e = p_k$. 
\end{proof}

At last we can prove the important Radon-Nikodym theorem, also
called the Lebesque decomposition.
\begin{proposition}
    Let $(X,\mathcal{A},\nu)$ be a $\sigma$-finite signed measure space. Let
    $\mu$ be a positive measure on $(X,\mathcal{A})$. Then there exists a
    unique decomposition,
    \begin{equation*}
        \nu(A) = \lambda(A) + \int_A f\;d\mu.
    \end{equation*}
    where $f \in \mathcal{L}^1(X,\mu)$ and $\lambda$ is a positive measure with $\lambda \bot \nu$.
\end{proposition}
\begin{proof}
    Assume first that $\nu(X) < \infty$ and $\nu$ is a positive measure.
    Let
    \begin{equation*}
        \mathcal{F} = \left\{f \;:\; \int_A f\;d\mu \leq \nu(A)\right\}.
    \end{equation*}
    Note that if $f,g \in \mathcal{F}$, then $\max\{f,g\} \in \mathcal{F}$.
    
    Let
    \begin{equation*}
        a = \sup\left\{\int_X f\;d\mu \;:\; f \in \mathcal{F}\right\}.
    \end{equation*}
    We have $a \leq \nu(X)$. Suppose that $\{g_n\}_{n=1}^\infty$
    is a sequence of measurable functions which approximate the supremum. That is,
    \begin{equation*}
        a = \lim_{n\rightarrow\infty} \int_X g_n\;d\mu.
    \end{equation*}
    
    We can choose $g_n$ to be an increasing sequence, since $\mathcal{F}$
    is closed under taking finite maximums. Then let $f = \lim_{n\rightarrow\infty} g_n$.
    
    Hence by the monotone convergence theorem we have
    \begin{equation*}
        \int_A f\;d\mu = \lim_{n\rightarrow\infty} \int_A g_n\;d\mu \leq \nu(A).
    \end{equation*}
    Hence $f \in \mathcal{F}$, and
    \begin{equation*}
        \int_X f\;d\mu = a.
    \end{equation*}
    
    
    Now define the measure $\lambda$,
    \begin{equation*}
        \lambda(A) = \nu(A) - \int_Af\;d\mu
    \end{equation*}
    $\lambda$ is a positive measure since $f \in \mathcal{F}$.
    
    If $\lambda$ is not mutually singular to $\mu$, then there exists $\varepsilon > 0$
    and an element $E \in \mathcal{A}$ such that $E$ is hereditarily positive for 
    $\lambda - \varepsilon \mu$ and $\mu(E) > 0$.
    
    Let $a \in A$. Then,
    \begin{equation*}
        \varepsilon\mu(E\cap A) \leq \lambda(A) = \nu(A) - \int_A f\;d\mu.
    \end{equation*}
    Then consider
    $f_\varepsilon := f+\varepsilon\chi_E$. We have $\int_X f_\varepsilon \;d\mu > \int_X f\;d\mu$.
    But this is impossible.
    
    Hence $\lambda \bot \mu$.
    
    Now we generalise to the case where $\nu$ is a $\sigma$-finite positive measure
    by finding an increasing sequence $\{A_k\}_{k=1}^\infty \subseteq \mathcal{A}$
    and finding a decomposition $d\lambda_k+f_kd\mu$ for the measure
    obtained by restricting $\nu$ to $A_k$. By taking limits, this
    gives us a decomposition for all of $X$.
    
    Now use the Jordan decomposition to increase to the case where $\nu$ is signed.    
\end{proof}

\section{Young's inequality}
Suppose that $(G,+)$ is a locally compact abelian group. The general
theory of locally compact abelian groups tells us that $G$ possesses
a unique (up to scaling) translation invariant measure
which is \emph{inner regular} and \emph{outer regular}. Denote
this measure by $\mu$. Given $f,g \in C_c(G)$, define
\begin{equation*}
    (f*g)(x) := \int_G f(x-y)g(y)\;d\mu(y).
\end{equation*}

\begin{proposition}
    Let $1 \leq p,q,r \leq \infty$ with $1+1/r = 1/p + 1/q$. Then if
    $f \in \mathcal{L}^p(G,\mu)$ and $g \in \mathcal{L}^q(G,\mu)$,
    then $f*g \in \mathcal{L}^r(G,\mu)$, with
    \begin{equation*}
        \|f*g\|_r \leq \|f\|_p\|g\|_q.
    \end{equation*}
\end{proposition}
\begin{proof}
    
\end{proof}

\end{document}
