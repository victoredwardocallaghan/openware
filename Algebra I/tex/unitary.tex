% Copyright © 2013 Edward O'Callaghan. All Rights Reserved.
% !Tex root = Algebra I.tex

\section{Unitary Operators} % (fold)
\label{sec:unitary}

A linear operator whose inverse is its adjoint is said to be
\emph{unitary}. Unitary operators can thus be thought of as
generalizations of complex numbers whose absolute value is
\emph{unity} and hence the name. More precisely,

\begin{defn}[Unitary]
 A linear map $U : \mathcal{V} \to \mathcal{V}$ is said to be
 \emph{unitary} if $U^{*} U = I = U U^{*}$.
\end{defn}

We thus have the subgroup of the general linear group $GL$
called the \emph{unitary group} defined as,
\[
 U_n(\C) = \{ U \in GL_n(\C) : U^{*}U = I = UU^{*} \}.
\]

\begin{cor}
 If $U$ is unitary then we have that adjoint of $U$
 acts as its inverse $U^{*}=U^{-1}$.
\end{cor}
