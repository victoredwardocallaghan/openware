% Copyright © 2013 Edward O'Callaghan. All Rights Reserved.
% !Tex root = Algebra I.tex

\section{Self-Adjoint Hermitian Operators} % (fold)
\label{sec:hermitian}

\begin{defn}[Hermitian Self-Adjoint]
 The linear map $T: \mathcal{V} \to \mathcal{V}$ is said to
 be \emph{self-adjoint}, or \emph{hermitian}, if
 \[
  \langle \mathbf{x}, T \mathbf{x} \rangle
  = \langle T \mathbf{x}, \mathbf{x} \rangle.
 \]
\end{defn}

\begin{lem}
 If $T$ is self-adjoint hermitian then $T^{*}=T$.
\end{lem}

\begin{rem}
 If the vector space $\mathcal{V}$ is defined over the reals
 then we have that $T=T^{T}$, that is $T$ is \emph{symmetric}.
\end{rem}

\begin{prop}
 Suppose $T:\mathcal{V} \to \mathcal{V}$ is hermitian self-adjoint.
 Then $T$ has all real eigenvalues.
\end{prop}

\begin{proof}
 By definition $T \mathbf{x} = \lambda \mathbf{x} : \mathbf{x} \neq \mathbf{0}$.
 Hence,
 \begin{align*}
  T \mathbf{x} &= T^{*} \mathbf{x}
  \\
  \Rightarrow \lambda \mathbf{x} &= \bar{\lambda} \mathbf{x}
  \\
  \Rightarrow \lambda &= \bar{\lambda}
  \\
  \Rightarrow \lambda & \in \R. \qedhere
 \end{align*}
\end{proof}

\begin{lem}
 If $T:\mathcal{V} \to \mathcal{V}$ is hermitian self-adjoint.
 Then the characteristic polynomial $\chi_{T}(\lambda)$ factors
 completely into linear factors over the reals $\R$.
\end{lem}

\begin{prop}
 Suppose that $T: \mathcal{V} \to \mathcal{V}$ is hermitian
 self-adjoint and let $\mathcal{W} \leq \mathcal{V}$ be a
 $T$-invariant subspace. Then $\mathcal{W}^{\perp}$ is $T$-invariant.
 Furthermore, if $\dim(\mathcal{W}) < \infty$ is finite dimensional
 then $\mathcal{V} = \mathcal{W} \oplus \mathcal{W}^{\perp}$ is a
 $T$-invariant direct sum.
\end{prop}

\begin{proof}
 Suppose that $\mathbf{x} \in \mathcal{W}$ and
 $\mathbf{x}' \in \mathcal{W}^{\perp}$ and $\mathcal{W} \leq \mathcal{V}$
 is a finite dimensional $T$-invariant subspace.
 Then $T \mathbf{x} \in \mathcal{W}$ and so,
 \begin{align*}
  \langle \mathbf{x}, T \mathbf{x}' \rangle &=
  \langle T \mathbf{x}, \mathbf{x}' \rangle = 0
  \\
  & \Rightarrow  T \mathbf{x}' \perp \mathcal{W}. \qedhere
 \end{align*}
\end{proof}
