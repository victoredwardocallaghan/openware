% Copyright © 2012 Edward O'Callaghan. All Rights Reserved.

\section{Jordan Canonical Form} % (fold)
\label{sec:JordanForm}

The Jordan (normal) canonical form is a set of "almost diagonal"
matries, called the $\emph{Jordan blocks}$. The Jordan blocks
contain the diagonal components composed of eigenvalues. We call
this form, the $\emph{block diagonal form}$. The set of Jordan
blocks form a $\emph{equivalence class}$ under similarity invariance
of the eigen-invariant subspaces.

\begin{defn}[Jordan block]
	A Jordan block $\mathcal{J}_k(\lambda)$ is a $k \times k$
	$\emph{upper triangular}$ square matrix of the form:
	\[
		\mathcal{J}_k(\lambda) =
		\begin{pmatrix}
			\lambda & 1 & & & \\
			0 & \lambda & 1 & & \text{\huge{0}} \\
			& & \ddots & \ddots & \\
			\text{\huge{0}} & & & \ddots & 1 \\
			& & & & \lambda
		\end{pmatrix}
	\]
	where there are $k-1$ (+1)'s in the $\emph{superdiagonal}$ and
	$k$ of a particular scalar eigenvalue $\lambda$ in the main
	diagonal while all other entries are zero. The trivial $1 \times 1$
	Jordan block of $\lambda$ is $\mathcal{J}_1(\lambda)=[\lambda]$.
\end{defn}

\begin{defn}[Jordan matrix]
	A Jordan matrix $\mathcal{J} \in \mathcal{M}_{n,n}(\F)$ is a
	\nameref{sec:directsum} of Jordan blocks. That is,
	\[
		\mathcal{J} =
		\bigoplus_{i=1}^{k} \mathcal{J}_i(\lambda_i)
	\]
	where each $\lambda_i \in \sigma(A)$ for some matrix $A \in \mathcal{M}(\F)$.
\end{defn}
