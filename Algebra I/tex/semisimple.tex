% Copyright © 2013 Edward O'Callaghan. All Rights Reserved.
% !Tex root = Algebra I.tex

\section{Semi-simple Operators} % (fold)
\label{sec:semi-simple}

We may extend the spectral theorem to normal operators.
In particular, normal operators are then characterised
by the spectral theorem. In the case of finite-dimensional
vector spaces over the algebraically closed field of complex
numbers we have that (compact) normal operators are then
unitarily diagonalisable.

\begin{defn}[Semi-simple Operator]
 Let $T : \mathcal{V} \to \mathcal{V}$ be a linear map on
 a finite vector space $\mathcal{V}$. Then $T$ is said to
 be a \emph{semi-simple operator} if and only if $T$ is
 a \emph{direct sum} of its eigen-subspaces.
\end{defn}

\begin{lem}
 Given the set $\{ T_i \}_i$ of semi-simple operators, we
 have that $\bigoplus_{i} T_i$ is a semi-simple opertator.
 Hence, the restriction of any semi-simple operator
 $T: \mathcal{V} \to \mathcal{V}$ on a $T$-invariant subspace
 $\mathcal{W} \leq \mathcal{V}$ in the following way,
 $T_{\mathcal{W}} : \mathcal{W} \to \mathcal{W}$, is a
 semi-simple operator.
\end{lem}

\begin{prop}
 A linear map $T : \mathcal{V} \to \mathcal{V}$ on a finite
 vector space $\mathcal{V}(\F)$ is diagonalisable if and only
 if $T$ is a semi-simple operator. Provided that the field $\F$
 is algebraically closed.
\end{prop}

\begin{proof}
 TODO..
\end{proof}

\begin{prop}
 Let $S: \mathcal{V} \to \mathcal{V}$ and $T: \mathcal{V} \to \mathcal{V}$
 be two linear operators defined on a finite vector space $\mathcal{V}$.
 Let $\mathcal{E}_{\lambda}^{S}$ and $\mathcal{E}_{\mu}^{T}$ denote the
 eigen invariant subspaces. Then, if the commutator of $S$ and $T$ is
 zero $[S, T] = S T - T S = 0$, we have that $\mathcal{E}_{\lambda}^{S}$
 is $T$-invariant and $\mathcal{E}_{\mu}^{T}$ is $S$-invariant.
\end{prop}

\begin{proof}
 Let $\mathcal{E}_{\lambda}^{s} = \ker(S - \lambda I)$ and note that
 $\mathcal{E}_{\lambda}^{s} \leq \mathcal{V}$. Then by definition we
 are required to show that
 $T(\mathcal{E}_{\lambda}^{S}) \subseteq \mathcal{E}_{\lambda}^{S}$
 for $T$-invariancy. Given that the commutator $[S,T]=0$ we see that,
 \begin{align*}
  T(\mathcal{E}_{\lambda}^{S})
  &= T(\{ \mathbf{x} \in \mathcal{V} : (S - \lambda I) \mathbf{x} = \mathbf{0} \})
  \\
  &= \ker(T(S - \lambda I))
  \\
  &= \ker(TS - \lambda T)
  \\
  &= \ker(ST - \lambda T) \tag{Since $ST=TS$}
  \\
  &= \ker((S - \lambda I)T)
  \\
  &= \{ \mathbf{x} \in \mathcal{V} : (S - \lambda I) T \mathbf{x} = \mathbf{0} \}
  \\
  & \subseteq \{ \mathbf{x} \in \mathcal{V} : (S - \lambda I) \mathbf{x} = \mathbf{0} \}
  = \mathcal{E}_{\lambda}^{S}. \qedhere
 \end{align*}
\end{proof}

\begin{thm}[Simultaneous diagonalisation]
 Let $X: \mathcal{V} \to \mathcal{V}$ and $Y: \mathcal{V} \to \mathcal{V}$
 be commuting semi-simple operators defined on some finite vector space
 $\mathcal{V}$. Let $\mathcal{E}_{\lambda_1}^{X}, \dots, \mathcal{E}_{\lambda_r}^{X}$
 and $\mathcal{E}_{\mu_1}^{Y}, \dots, \mathcal{E}_{\mu_s}^{Y}$ denote the eigen
 invariant subspaces respectively. Then,
 \[
  \mathcal{V} = \bigoplus_{i=1}^{r} \bigoplus_{j=1}^{s}
  \mathcal{E}_{\lambda_i}^{X} \cap \mathcal{E}_{\mu_j}^{Y}.
 \]
\end{thm}

\begin{proof}
 TODO..
\end{proof}

\begin{defn}[Normal Operator]
 A linear map $T : \mathcal{V} \to \mathcal{V}$ on the vector space $\mathcal{V}(\C)$
 is said to be \emph{normal} if the commutator $[T, T^{*}]=0$.
\end{defn}

\begin{thm}[Spectral Theorem for Normal Operators]
 A linear operator $T : \mathcal{V} \to \mathcal{V}$ defined on a finite vector space
 $\mathcal{V}(\C)$ is \emph{unitarily diagonalisable} if and only if it is \emph{normal}.
\end{thm}

\begin{proof}
 Consider the linear operator $T : \mathcal{V} \to \mathcal{V}$ with $\dim \mathcal{V} < \infty$.
 Let $U : \C^n \to \mathcal{V}$ be unitary and so orthonormal and let $\Lambda = U^{-1} T U$ be
 diagonal. Since then $U$ is unitary we that $U^{-1}=U^{*}$ and since $\Lambda$ is diagonal we
 have that the commutator is $[\Lambda,\Lambda^{*}]=0$. Hence,
 \begin{align*}
  T T^{*} &= (U \Lambda U^{*}) (U \Lambda U^{*})^{*}
  \\
  &= (U \Lambda U^{*}) (U \Lambda^{*} U^{*})
  \\
  &= U (\Lambda \Lambda^{*}) U^{*}
  \\
  &= U (\Lambda^{*} \Lambda) U^{*}
  \\
  &= (U \Lambda^{*} U^{*}) (U \Lambda U^{*})
  \\
  &= (U \Lambda U^{*})^{*} (U \Lambda U^{*})
  = T^{*} T
  \\
  & \Rightarrow [T,T^{*}] = 0.
 \end{align*}
 Conversaly, suppose now that $[T,T^{*}]=0$. TODO.. Finish proof..
\end{proof}

\begin{lem}
 Let $T: \mathcal{V} \to \mathcal{V}$ be a linear operator on the finite vector space
 $\mathcal{V}(\C)$. Let $X= \frac{1}{2}(T+T^{*})$ and $Y=\frac{1}{2 i}(T - T^{*})$.
 Then both $X$ and $Y$ are hermitian operators such that $T=X + i Y$ and if $T$ is normal
 then $X$ and $Y$ commute $[X,Y]=0$. The operator $X$ is said to be the self-adjoint part
 of $T$ and the operator $Y$ is the (anti)self-adjoint part of $T$.
\end{lem}

\begin{proof}
 By the spectral theorem for self-adjoint operators, $X$ and $Y$ are semi-simple operators.
 Suppose then that $\mathcal{E}_{\lambda_1}^{X}, \dots, \mathcal{E}_{\lambda_r}^{X}$
 and $\mathcal{E}_{\mu_1}^{Y}, \dots, \mathcal{E}_{\mu_s}^{Y}$ denote the eigen
 invariant subspaces respectively. Then we have that they are pairwise mutually orthogonal.
 By the simultaneous diagonalisation theorem we can find an orthonormal coordinate system
 $U : \C^n \to \mathcal{V}$ to the direct sum orthogonal decomposition,
 \[
  \mathcal{V} = \bigoplus_{i=1}^{r} \bigoplus_{j=1}^{s}
  \mathcal{E}_{\lambda_i}^{X} \cap \mathcal{E}_{\mu_j}^{Y}.
 \]
 such that we have both $\Lambda_{X} = U^{*} X U$ and $\Lambda_{Y} = U^{*} Y U$ as diagonal.
 Hence,
 \begin{align*}
 U^{*} T U &= U^{*} (X + i Y) U
 \\
 &= U^{*} X U + i U^{*} Y U
 \\
 &= \Lambda_X + i \Lambda_Y
 \end{align*}
 which is diagonal.
\end{proof}
