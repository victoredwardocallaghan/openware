% Copyright © 2013 Edward O'Callaghan. All Rights Reserved.
% !Tex root = Algebra I.tex

\section{Semi-simple Operators} % (fold)
\label{sec:semi-simple}

We may extend the spectral theorem to normal operators.
In particular, normal operators are then characterised
by the spectral theorem. In the case of finite-dimensional
vector spaces over the algebraically closed field of complex
numbers we have that (compact) normal operators are then
unitarily diagonalisable.

\begin{defn}[Semi-simple Operator]
 Let $T : \mathcal{V} \to \mathcal{V}$ be a linear map on
 a finite vector space $\mathcal{V}$. Then $T$ is said to
 be a \emph{semi-simple operator} if and only if $T$ is
 a \emph{direct sum} of its eigen-subspaces.
\end{defn}

\begin{lem}
 Given the set $\{ T_i \}_i$ of semi-simple operators, we
 have that $\bigoplus_{i} T_i$ is a semi-simple opertator.
 Hence, the restriction of any semi-simple operator
 $T: \mathcal{V} \to \mathcal{V}$ on a $T$-invariant subspace
 $\mathcal{W} \leq \mathcal{V}$ in the following way,
 $T_{\mathcal{W}} : \mathcal{W} \to \mathcal{W}$, is a
 semi-simple operator.
\end{lem}

\begin{prop}
 A linear map $T : \mathcal{V} \to \mathcal{V}$ on a finite
 vector space $\mathcal{V}(\F)$ is diagonalisable if and only
 if $T$ is a semi-simple operator. Provided that the field $\F$
 is algebraically closed.
\end{prop}

\begin{proof}
 TODO..
\end{proof}
