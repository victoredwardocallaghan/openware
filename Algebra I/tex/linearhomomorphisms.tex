% Copyright © 2012 Edward O'Callaghan. All Rights Reserved.

\section{Vector space Homomorphisms} % (fold)
\label{sec:linearmaps}

\begin{defn}[Vector space Homomorphism]
	A vector space homomorphism is the structure preserving mapping between two vector spaces.
	That is, for some vector spaces $\mathcal{U}$ and $\mathcal{V}$ over common field $\F$
	with morphism $\varphi : \mathcal{U} \to \mathcal{V}$ we have,
	\[
		\varphi(u_1 + \lambda u_2) = \varphi(u_1) + \lambda \varphi(u_2) : u_1,u_2 \in \mathcal{U} \text{ and } \lambda \in \F.
	\]
\end{defn}

\begin{rem}
	A vector space homomorphism preserves linearity and so we typically call the
	homomorphism a linear map.
\end{rem}

% Kernel
\begin{defn}[Kernel]
	Consider the linear map $\varphi : \mathcal{U} \to \mathcal{V}$, the
	$\emph{kernel}$ of $\varphi$, denoted $\ker(\varphi)$ is given by:
	\[
		\ker(\varphi) = \{ u \in \mathcal{U} : \varphi(u) = \mathbf{0} \in \mathcal{V} \}.
	\]
	The set or space of elements inside the kernel is thus called the $\emph{null space}$.
\end{defn}

Notice that the null space is a subspace of the linear map where any elements inside this
subspace are taken to the trivial vector space of $\mathcal{V} = \mathbf{0}$ under the linear
map.

\begin{exmp}
	Consider the linear map $T:\R^3 \to \R^2$ where,
	\[
		T=
		\begin{pmatrix}
			1 & 3 & 2 \\
			1 & 2 & 5 
		\end{pmatrix}.
	\]
	To find the kernel, $\ker(T)$, we must use the definition and hence solve,
	$T(\mathbf{x}) = \mathbf{0}$ for some $\mathbf{x} \in \R^3$. Hence, by Gaussian
	elimination augmenting $(T | \mathbf{0})$, we have;
	\[
		\ker(T) = \{ \lambda \begin{pmatrix} -11 \\ 3 \\ 1 \end{pmatrix} : \lambda \in \F \}
	\]
\end{exmp}
% TODO ..

% Nullity
\begin{defn}[Nullity]
	The $\emph{nullality}$ is the dimension of the null space or kernel.
\end{defn}

\begin{rem}
	Note that the nullity gives a measure of the injectivity of the linear map
	$\varphi$.
\end{rem}

% Image
\begin{defn}[Image]
	For some morphism $\varphi : \mathcal{U} \to \mathcal{V}$ the $\emph{image}$ is defined as,
	\[
		img(\varphi) =
		\varphi(\mathcal{U}) = \{ v \in \mathcal{V} : v = \varphi(u) \text{for some} u \in \mathcal{U} \}.
	\]
\end{defn}

% Preimage
\begin{defn}[Preimage]
	For some morphism $\varphi : \mathcal{U} \to \mathcal{V}$ the $\emph{preimage}$ is defined as,
	\[
		\stackrel{\leftarrow}{\varphi}(\mathcal{V}) = \{ u \in \mathcal{U} : \varphi(u) \in \mathcal{V} \}
	\]
\end{defn}

\begin{rem}
	The preimage can be defined even when no inverse morphism exists.
\end{rem}

% Rank-Nullality Theorm
\begin{thm}
	Let $\mathcal{U}$ and $\mathcal{V}$ be vector spaces over some common field $\F$
	and some linear map $\varphi : \mathcal{U} \to \mathcal{V}$ then
	\[
		\dim( img(\varphi) ) + \dim( \ker(\varphi) ) = \dim( \mathcal{U} )
	\]
\end{thm}

% Monomorphism
\begin{defn}[Injective]
	If, for the linear map $\varphi : \mathcal{U} \to \mathcal{V}$, we have
	\[
		\varphi(u_1) = \varphi(u_2) \implies u_1 = u_2 \, \forall u_1,u_2 \in \mathcal{U}
	\]
	then $\varphi$ is said to be $\emph{injective}$.
\end{defn}

% Epimorphism
\begin{defn}[Surjective]
	If, for the linear map $\varphi : \mathcal{U} \to \mathcal{V}$, we have
	\[
		\forall v \in \mathcal{V} \exists u \in \mathcal{U} : \varphi(u) = v
	\]
	then $\varphi$ is said to be $\emph{surjective}$.
\end{defn}

% Isomorphism 
\begin{defn}[Bijective]
	If the linear map $\varphi : \mathcal{U} \to \mathcal{V}$ is both
	injective and subjective then $\varphi$ is said to be $\emph{bijective}$, or,
	isomorphic in the context vector space homomorphisms.
\end{defn}

\subsection{Properties of linear maps}
One may note that a linear map can always be represented in the form of a matrix and as such
the determinant can be taken given that linear maps are from vector spaces to vector spaces
over a common field.
Hence, the conceptual question arises, geometrically, what exactly is the determinant? It
turns out the determinant is a $\emph{measure}$ of dilation much like how scaler multiplies
of the unit scalar dilate it. That is, $3*1=3$ is the dilation of the unit scalar $1$ by a
measure of $3$. Now, recall that a field is a vector space over itself and so we may use this
to generalise our intuition from above. Thus, we may view a linear map as a linear dilation
of some vector in one vector space to another vector space and the determinant as the measure
of dilation of this mapping.

TODO.. motivate to prove the determinant definition here.. and motiave to the charteristic polynomial
and eigen spaces..
