\section{Algebras} % (fold)
\label{sec:algebras}

\begin{defn}[Algebras]
	A finite-dimensional (associative) $\emph{algebra}$ $\mathcal{A}$
	over a field $\mathbb{F}$ is a finite-dimensional vector space over
	$\mathbb{F}$ equipped with a law of composition, that is, a mapping
	(multiplication) $(a,b) \to ab$ from $\mathcal{A} \times \mathcal{A}$
	into $\mathcal{A}$ which satisfies
	\begin{itemize}
		\item $(ab)c=a(bc)$ (associativity),
		\item $a(b + c) = ab + ac$,
		\item $(a + b)c = ac + bc$,
		\item $\lambda (ab) = (\lambda a)b = a(\lambda b)$,
	\end{itemize}
	for $\lambda \in \mathbb{F}$ and $a,b,c \in \mathcal{A}$.
\end{defn}

\begin{rem}
	A algebra $\mathcal{A}$ is called $\emph{unital}$ if there exists
	$1 \in \mathcal{A}$, the $\emph{idenity element}$, such that
	$1a=a1=a \, \forall a \in \mathcal{A}$.
\end{rem}

\begin{rem}
	A algebra $\mathcal{A}$ is $\emph{commutative}$ if
	$ab=ba \, \forall a,b \in \mathcal{A}$.
\end{rem}
